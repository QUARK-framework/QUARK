\documentclass[12pt, ngerman]{article}
\usepackage{import}
\usepackage{preamble}
\usepackage{booktabs}
\addbibresource{literatur.bib}



\begin{document}

    % Large images on titel page
    \lhead{\includegraphics[height=2cm]{images/logo_tex.png}}
    \chead{}
    \rhead{\includegraphics[height=2cm]{images/logo_optware.png}}
    \lfoot{\includegraphics[width=9cm]{images/logo_optware_supplement_and_claim_rgb_1258x71px_300dpi.jpg}}
    \cfoot{}
    \rfoot{\thepage}
    \renewcommand{\headrulewidth}{0pt}

    % Title page
    %--------------------------------------
    \title{BenchQC - Anwendungsfälle der Optimierung}
    \date{\today}
    \author{OptWare GmbH}
    \maketitle
    \thispagestyle{fancy}
    %--------------------------------------

    % Abstract
    %--------------------------------------
    \begin{abstract}
        Das Forschungsprojekt BenchQC, Anwendungsgetriebenes Benchmarking von
        Quantencomputern, befasst sich mit echten Anwendungsfällen der Industrie
        aus Machine Learning, Simulation und Optimierung.
        In diesem Dokument sollen die folgenden industriellen Anwendungsfelder der Optimierung
        betrachtet werden:
        \begin{itemize}
            \item Fließbandabstimmung
            \item Reihenfolgeplanung
            \item Testfahrzeugkonfiguration
            \item ...
        \end{itemize}
        Die einzelnen Anwendungsfälle werden zunächst definiert,
        ihre klassischen Algorithmen erarbeitet und anschließend
        für die Anwendung auf Quantencomputern modifiziert.
        Abschließend werden die klassischen und QC Lösungen mithilfe eines
        Benchmarking Frameworks (QUARK) verglichen.
    \end{abstract}
    \newpage
    %--------------------------------------

    % Different header images after title page
    %--------------------------------------
    \setlength{\headheight}{1.5cm}
    \lhead{}
    \chead{}
    \rhead{\includegraphics[height=1.2cm]{images/logo_optware.png}}
    \pagestyle{fancy}
    %--------------------------------------

    % Table of contents
    %--------------------------------------
    \newpage
    \tableofcontents
    %--------------------------------------

    % Sections
    %--------------------------------------
    \newpage
    \section{Anwendungsfälle}

Im Folgenden sollen zunächst die industriellen Anwendungsfälle vorgestellt werden.

\subsection{Fließbandfertigung}\label{subsec:fliessbandfertigung}

Moderne Produktion, wie zum Beispiel die Automobilproduktion, erfolgt oft mithilfe der Fließbandfertigung.
Bei der Fließbandfertigung durchläuft das zu fertigende Produkt auf dem Weg zur Fertigstellung zahlreiche
Arbeitsstationen, in denen die einzelnen Arbeitsschritte zur Fertigstellung des Produktes ausgeführt werden.

Die Beplanung einer Fertigungslinie erfolgt dabei in zwei Schritten:
Die Fließbandabstimmung (cf.\ \cref{subsec:anwendungsfall-fliessbandabstimmung}) und die Reihenfolgeplanung
(cf.\ \cref{subsubsec:anwendungsfall-reihenfolgeplanung}).

Die Fließbandabstimmung (\emph{Assembly Line Balancing}) beschäftigt sich damit,
Arbeitsschritte nach ihrem mittleren Zeitbedarf auf Arbeitsstationen zu verteilen.
In der Reihenfolgeplanung (\emph{Sequencing}) werden konkrete Folgen von Aufträgen gebildet,
die in Form einer operativen Feinplanung zu große Abweichung von den in der
Fließbandabstimmung verwendeten Mittelwerten auf den einzelnen Stationen verhindern sollen.
Die Fließbandabstimmung liefert somit eine taktische Grobplanung,
auf der die operative Reihenfolge(fein)planung basiert.

Der mittlere Zeitbedarf der Arbeitsschritte, die als Ausgangsbasis der Fließbandabstimmung dient,
ergibt sich aus dem tatsächlichen Zeitbedarf für einen Arbeitsschritt multipliziert
mit der erwarteten Häufigkeit des Auftretens dieses Schrittes.
Dauert z.B.\ die Montage einer Anhängerkupplung 100 Sekunden,
und ist eine Anhängerkupplung nur bei jedem zehnten Fahrzeug zu erwarten,
beträgt der mittlere Zeitbedarf 10 Sekunden.

\subsubsection{Anwendungsfall Reihenfolgeplanung (Sequencing)}\label{subsubsec:anwendungsfall-reihenfolgeplanung}
% Allgemeine (high-level) Beschreibung:
Wie bereits in~\cref{subsec:fliessbandfertigung} beschrieben,
folgt in modernen Produktionen auf die Fließbandabstimmung,
in der die benötigten Arbeitsschritte auf Arbeitsstationen verteilt wurden,
häufig die operative Reihenfolge(fein)planung, auch \emph{Sequencing} genannt.
Hierbei geht es nun darum, konkrete Arbeitsaufträge,
die die einzelnen Stationen durchlaufen müssen, in die optimale Reihenfolge zu bringen.
Wir legen hier eine sogenannte \emph{mixed-model assembly line} zugrunde,
was bedeutet, dass die Aufträge die Herstellung verschiedener Modelle eines Basisprodukts beschreiben.
Die benötigten Arbeitsschritte und die Zeit für deren Erledigung unterscheiden sich somit pro Auftrag.
Die operative Feinplanung muss daher u.a.\ dafür sorgen, dass zu keinem Zeitpunkt zu große Abweichungen (Drift)
von den in der Fließbandabstimmung verwendeten Mittelwerten pro Station entstehen.

Eine Probleminstanz für das Sequencing besteht aus
einem Fließband, einer Liste an Aufträgen und einer Liste an Anforderungen bzw.\ Regeln für die Reihenfolge.
Das Ziel ist es, die Aufträge in eine Reihenfolge zu bringen, die diese Regeln beachtet und eine vorher definierte Zielfunktion optimiert.
Im Folgenden werden nun die Eingabeparameter und die verschiedenen Komponenten der Zielfunktion und einer validen Lösung genauer erläutert.

\begin{figure}[H]
    \begin{subfigure}[c]{0.48\textwidth}
        \centering
        \tikzfig{images/reihenfolgeplanung/assembly_line}
        \caption{%
        \label{fig:sequencing_assembly_line}
        Ein Fließband mit drei Stationen,
            die jeweils aus einem Arbeitsplatz bestehen,
            einen Takt dauern
            und eine maximale Driftzeit erlauben.
        Ein Auftrag an Station 1 muss vollständig abgearbeitet sein,
            bevor er an Station 2 weiterbearbeitet werden kann (durchgezogene Linie).
        }
    \end{subfigure}%
    \hfill
    \begin{subfigure}[c]{0.48\textwidth}
        \centering
        \tikzfig{images/reihenfolgeplanung/auftraege}
        \caption{%
        \label{fig:sequencing_auftraege}
        Die Aufträge symbolisieren durch die verschiedenen Formen
        die Herstellung verschiedener Varianten eines Basisprodukts
        und haben eine individuelle Bearbeitungszeit pro Station.
        Die zu erledigenden Aufträge sind vorgegeben.
        }
    \end{subfigure}
    \caption{%
        \label{fig:sequencing_instance}
        Eine einfache Probleminstanz für das Sequencing.
        Das Ziel ist eine optimale Abarbeitungsreihenfolge für die Auftragsliste zu finden.
    }
\end{figure}

\paragraph{Das Fließband}
ist üblicherweise bereits durch die Fließbandabstimmung (siehe~\cref{subsec:anwendungsfall-fliessbandabstimmung})
\marginpar{\scriptsize {\emph{Fließband}:\\ \phantom{s} Taktzeit \\ \phantom{s} Stationen}}
konfiguriert worden und hat somit eine feste Anzahl an \emph{Arbeitsstationen}---denen die verschiedenen Arbeitsschritte,
die an einem Produkt ausgeführt werden müssen, zugeteilt wurden---und eine feste \emph{Taktzeit}.
Ein einfaches Beispiel für ein Fließband mit drei Stationen ist in~\cref{fig:sequencing_assembly_line} illustriert.

Jede Station hat eine vordefinierte Anzahl an \emph{Takten}, für die ein Produkt auf dieser Station verweilen kann,
\marginpar{\scriptsize {\emph{Station}:\\ \phantom{s} Takte \\ \phantom{s} Plätze}}
und ist selbst wieder in eine Menge an \emph{Arbeitsplätze} aufgeteilt.
Jeder Arbeitsplatz kann das Produkt selbst wieder für eine feste Anzahl an Takten bearbeiten
(was auch dazu führen kann, dass sich ein Arbeitsplatz über mehrere Stationen erstreckt)
\marginpar{\scriptsize {\emph{Arbeitsplatz}:\\ \phantom{s} Takte \\ \phantom{s} Vorgänger \\ \phantom{s} Drifts}}
und hat eine Liste an Vorgängerarbeitsplätzen, an denen ein Produkt schon vollständig bearbeitet worden sein muss,
bevor es an dem aktuellen Arbeitsplatz bearbeitet werden kann. % besteht keine Vorgängerbeziehung zwischen zwei Arbeitsplätzen,
% kann also gleichzeitig am Produkt gearbeitet werden (wenn z.B. ein Arbeitsplatz der vorherigen Station über diese hinaus arbeitet)

Des Weiteren können für jeden Arbeitsplatz erlaubte \emph{Drifts} festgelegt werden:
Die Arbeit an diesem Platz darf dann um diese Zeiteinheit früher beginnen bzw.\
über das Ende der Taktzeit hinausdauern.
Bei der ersten Arbeitsstation kann es zudem zu einem initialen Drift kommen,
der z.B.\ durch Verspätungen aus einer vorangegangenen Schicht (mit eigener Reihenfolgeplanung) entstanden ist.
Die Driftgrenzen dürfen nie überschritten werden,
da das Produkt dann beispielsweise außerhalb des Arbeitsradius bestimmter Werkzeuge gelangt und das Fließband gestoppt werden müsste.
Wenn der maximale Drift an einem Arbeitsplatz überschritten würde, muss daher eine sogenannte \emph{Springerin} eingesetzt werden,
die dann pünktlich den Folgeauftrag übernimmt, während die Person an dem Arbeitsplatz ihren aktuellen Auftrag beendet.
Diese Person kann dann den Folgeauftrag, der ja bereits von der Springerin bearbeitet wird,
überspringen und pünktlich zum Taktschlag mit dem nächsten Auftrag fortfahren.

\paragraph{Die Aufträge} bestehen aus einer Auflistung verschiedener Arbeitsschritte, die auf den Arbeitsstationen ausgeführt werden müssen,
\marginpar{\scriptsize {\emph{Auftrag}:\\ \phantom{s} Zeit \\ \phantom{s} Attribute}}
um ein konkretes Produkt herzustellen.
Pro Auftrag kennen wir die benötigte \emph{Arbeitszeit} pro Arbeitsplatz und eine Liste von \emph{Attributen},
anhand derer Regeln für die Reihenfolge bestimmt werden können.
\todo{\tiny Sollen Attribute zunächst vernachlässigt werden? Wenn nicht, sollte das Beispiel ggf. erweitert werden}
Ein vereinfachtes Beispiel ohne Attribute ist in~\cref{fig:sequencing_auftraege} dargestellt.

\paragraph{Die Regeln} schränken den Lösungsraum, also die Anzahl zulässiger Reihenfolgen für die Aufträge,
zusätzlich ein.
So lässt sich beispielsweise festlegen, dass Aufträge mit einem bestimmten Attribut einen gewissen zeitlichen Mindestabstand haben.
Die Regeln stellen jedoch keine starren Vorschriften dar, sondern sollen so gut es geht erfüllt werden.
% ,weshalb sich ihre Nichterfüllung negativ auf den Zielfunktionswert auswirkt.

\paragraph{Die Lösung} einer konkreten Probleminstanz beinhaltet eine hinsichtlich der Zielfunktion optimierte Reihenfolge der Aufträge.
Konkret setzt sich der Zielfunktionswert aus drei Komponenten zusammen, die je nach Wichtigkeit mit einem festgelegten Faktor in die Zielfunktion eingehen:
Die Anzahl an benötigten Springerinnen, der Drift und die Anzahl an verletzen Regeln.
Im einfachen Beispiel (ohne Attribute und Regeln) in~\cref{fig:example_jumper} würden dementsprechend die benötigte Springerin
und die Drifts an Station 1 und 2 in die Zielfunktion eingehen.
\cref{fig:example_optimal} stellt zudem eine Lösung der Instanz dar, die ohne Springerinnen auskommt.

\begin{figure}[H]
    \centering
    \begin{subfigure}[t]{0.48\textwidth}
        \resizebox{1\textwidth}{!}{
            \tikzfig{images/reihenfolgeplanung/example_jumper}
        }
        \caption{%
        \label{fig:example_jumper}
        Die hier angegebene (nicht optimale) Lösung erfordert eine Springerin (rot),
        da der dritte Auftrag sonst den zulässigen maximalen Drift an Station 1 überschritte.
        }
    \end{subfigure}%
    \hfill
    \begin{subfigure}[t]{0.48\textwidth}
        \resizebox{1\textwidth}{!}{
            \tikzfig{images/reihenfolgeplanung/example_optimal}
        }
        \caption{%
        \label{fig:example_optimal}
        Diese Lösung ist hinsichtlich der Anzahl der Springerinnen und der aufsummierten Drifts optimal.
        }
    \end{subfigure}
    \caption{%
    \label{fig:example_sequencing}
    Beispiel für zwei Lösungen der Instanz aus \cref{fig:sequencing_instance}.
    Die festgelegte Auftragsreihenfolge liest sich von oben nach unten
    und die Bearbeitungszeiten der Aufträge sind durch die Länge der horizontalen Balken symbolisiert.
    Zudem markieren die gestrichelten Linien die maximalen Drifts an den Stationen.
    Da die Arbeit an Station 2 erst nach dem Arbeitsende an Station 1 beginnen kann,
    kann ein Drift an Station 1 den Bearbeitungsstart an Station 2 verzögern.
    Der Startzeitpunkt auf Station 2 ergibt sich also aus dem Maximum von dem Ende des Auftrags auf Station 1
    und dem Drift des vorherigen Auftrags auf Station 2.
    Station 3 hingegen hat keine Vorgängerbeziehungen und kann alle Aufträge pünktlich starten,
    weil sich ein Drift auf Station 2 nicht auf Station 3 auswirkt,
    sondern die Arbeiten an diesen Stationen gleichzeitig erledigt werden können.}
\end{figure}

% Input Parameter:
%Eine Probleminstanz für die Reihenfolgeplanung enthält folgende Informationen:
%\begin{itemize}
%    \item Für das Fließband (\textit{line}) gibt es:
%    \begin{itemize}
%        \item eine Liste an (verschiedenen) Arbeitsstationen (\textit{stations}) mit:
%        \begin{itemize}
%            \item Anzahl Takte (\textit{cycles}) %ermöglicht unterschiedliche Taktzeiten für die Stationen trotz globaler cycle time
%            \item einer Liste an (verschiedenen) Arbeitsplätzen (\textit{workplaces}) mit:
%            \begin{itemize}
%                \item Anzahl Takte (\textit{cycles})
%                \item Liste an Vorgängern (\textit{predecessors})
%                \item Informationen über den Drift (\textit{maxEarlyDrift, maxLateDrift, initialDrift})
%            \end{itemize}
%        \end{itemize}
%        \item eine globale Taktzeit (\textit{cycle time}) pro Arbeitsstation (wobei eine Station jedoch mehrere Takte dauern kann)
%    \end{itemize}
%    \item Außerdem gibt es eine Anzahl an Aufträgen (\textit{numOrders}), welche aus verschiedenen Arbeitsschritten,
%    die bei der Fließbandabstimmung (siehe \cref{subsec:anwendungsfall-fliessbandabstimmung}) beachtet wurden und
%    hier nicht mehr relevant sind.
%    \item Für jeden Auftrag (\textit{order}) kennen wir:
%    \begin{itemize}
%        \item die benötigte Arbeitszeit (\textit{processingTime}) an einer gegebenen Arbeitsstation
%        \item eine Liste an Attributen (\textit{Attributes})  % kein input -> automatisch erstellt
%    \end{itemize}
%    \item Zudem gibt es diverse Anforderungen an bzw.\ Regeln für die Reihenfolge
    %https://gitlab.optware.de/sequencer/product/90200438-sequencer-mvp/-/blob/main/libs/interfaces/src/assets/problem/problem_data_schema.json?ref_type=heads
%\end{itemize}

% Output -> result.svg
%Eine Lösung, also eine konkrete Reihenfolgeplanung, enthält
%\begin{itemize}
%    \item eine Reihenfolge der Aufträge (\textit{SequenceResult})
%    \item und einen Zielfunktionswert, der sich zusammensetzt aus:
%    \begin{itemize}
%        \item jumper
%        \item sequencingRules % Anzahl verletzter Regeln
%        \item lateDrift
%    \end{itemize}
%\end{itemize}

\subsubsection*{Datensatz}\label{subsubsec:datensatz-sequencer}
% https://gitlab.optware.de/sequencer/product/90200438-sequencer-mvp/-/blob/main/libs/interfaces/src/assets/data-generation/generation_input_scheme.json?ref_type=heads
Der Datensatz wurde mittels eines \textit{data generators} künstlich erzeugt.
Dafür wurden für die verschiedenen Parameter Werte ausgewählt bzw.\ ein Bereich angegeben, aus dem der
Wert des Parameters gleichverteilt ausgewählt wurde, um Instanzen verschiedener realer Probleme zu simulieren.

Um die Komplexität des Problems zu verringern, wurden in Bezug auf das oben präsentierte Problem folgende
Vereinfachungen vorgenommen:
\todo[inline]{Liste von Vereinfachungen updaten!}
\begin{itemize}
    \item Jede Station besteht aus nur einem Arbeitsplatz,
    d.h.\ die Aufteilung einer Arbeitsstation in verschiedene Arbeitsplätze wurde weggelassen.
    \item \dots
\end{itemize}
\todo[inline]{Nachdem die Parameter für den Datengenerator festgelegt wurden, sollte hier angegeben werden,
    wie die Parameter gesetzt wurden und wie viele Instanzen welcher Größe generiert wurden.
    Vielleicht kann der generierte Datensatz mit Grafiken/Tabellen dargestellt werden (wie viele Instanzen welcher Art, etc.)?!}
% für folgende Werte kann entweder eine Zahl oder eine Verteilung (number-or-distribution) angegeben werden:
    % cycles
    % numPredecessors
    % numWorkplaces
    % numStations
    % workloadDistribution (pro StationId -> aus dieser Verteilung wird gesamplet, um die processing time pro station zu ermitteln)
% für folgende Werte kann entweder eine Zeiteinheit oder eine Verteilung (time-unit-or-distribution) angegeben werden:
    % maxLateDrift
        % sind keine Drift Parameter angegeben, dann ist an dieser Workstation kein Drift erlaubt
    % cycleTime
%TODO: die Attribute werden automatisch erstellt. An welcher Stelle passiert das? Und wie?
\subsubsection{Anwendungsfall Reihenfolgeplanung (Sequencing)}\label{subsubsec:anwendungsfall-reihenfolgeplanung}
% Allgemeine (high-level) Beschreibung:
Wie bereits in~\cref{subsec:fliessbandfertigung} beschrieben,
folgt in modernen Produktionen auf die Fließbandabstimmung,
in der die benötigten Arbeitsschritte auf Arbeitsstationen verteilt wurden,
häufig die operative Reihenfolge(fein)planung, auch \emph{Sequencing} genannt.
Hierbei geht es nun darum, konkrete Arbeitsaufträge,
die die einzelnen Stationen durchlaufen müssen, in die optimale Reihenfolge zu bringen.
Wir legen hier eine sogenannte \emph{mixed-model assembly line} zugrunde,
was bedeutet, dass die Aufträge die Herstellung verschiedener Modelle eines Basisprodukts beschreiben.
Die benötigten Arbeitsschritte und die Zeit für deren Erledigung unterscheiden sich somit pro Auftrag.
Die operative Feinplanung muss daher u.a.\ dafür sorgen, dass zu keinem Zeitpunkt zu große Abweichungen (Drift)
von den in der Fließbandabstimmung verwendeten Mittelwerten pro Station entstehen.

Eine Probleminstanz für das Sequencing besteht aus
einem Fließband, einer Liste an Aufträgen und einer Liste an Anforderungen bzw.\ Regeln für die Reihenfolge.
Das Ziel ist es, die Aufträge in eine Reihenfolge zu bringen, die diese Regeln beachtet und eine vorher definierte Zielfunktion optimiert.
Im Folgenden werden nun die Eingabeparameter und die verschiedenen Komponenten der Zielfunktion und einer validen Lösung genauer erläutert.

\begin{figure}[H]
    \begin{subfigure}[c]{0.48\textwidth}
        \centering
        \tikzfig{images/reihenfolgeplanung/assembly_line}
        \caption{%
        \label{fig:sequencing_assembly_line}
        Ein Fließband mit drei Stationen,
            die jeweils aus einem Arbeitsplatz bestehen,
            einen Takt dauern
            und eine maximale Driftzeit erlauben.
        Ein Auftrag an Station 1 muss vollständig abgearbeitet sein,
            bevor er an Station 2 weiterbearbeitet werden kann (durchgezogene Linie).
        }
    \end{subfigure}%
    \hfill
    \begin{subfigure}[c]{0.48\textwidth}
        \centering
        \tikzfig{images/reihenfolgeplanung/auftraege}
        \caption{%
        \label{fig:sequencing_auftraege}
        Die Aufträge symbolisieren durch die verschiedenen Formen
        die Herstellung verschiedener Varianten eines Basisprodukts
        und haben eine individuelle Bearbeitungszeit pro Station.
        Die zu erledigenden Aufträge sind vorgegeben.
        }
    \end{subfigure}
    \caption{%
        \label{fig:sequencing_instance}
        Eine einfache Probleminstanz für das Sequencing.
        Das Ziel ist eine optimale Abarbeitungsreihenfolge für die Auftragsliste zu finden.
    }
\end{figure}

\paragraph{Das Fließband}
ist üblicherweise bereits durch die Fließbandabstimmung (siehe~\cref{subsec:anwendungsfall-fliessbandabstimmung})
\marginpar{\scriptsize {\emph{Fließband}:\\ \phantom{s} Taktzeit \\ \phantom{s} Stationen}}
konfiguriert worden und hat somit eine feste Anzahl an \emph{Arbeitsstationen}---denen die verschiedenen Arbeitsschritte,
die an einem Produkt ausgeführt werden müssen, zugeteilt wurden---und eine feste \emph{Taktzeit}.
Ein einfaches Beispiel für ein Fließband mit drei Stationen ist in~\cref{fig:sequencing_assembly_line} illustriert.

Jede Station hat eine vordefinierte Anzahl an \emph{Takten}, für die ein Produkt auf dieser Station verweilen kann,
\marginpar{\scriptsize {\emph{Station}:\\ \phantom{s} Takte \\ \phantom{s} Plätze}}
und ist selbst wieder in eine Menge an \emph{Arbeitsplätze} aufgeteilt.
Jeder Arbeitsplatz kann das Produkt selbst wieder für eine feste Anzahl an Takten bearbeiten
(was auch dazu führen kann, dass sich ein Arbeitsplatz über mehrere Stationen erstreckt)
\marginpar{\scriptsize {\emph{Arbeitsplatz}:\\ \phantom{s} Takte \\ \phantom{s} Vorgänger \\ \phantom{s} Drifts}}
und hat eine Liste an Vorgängerarbeitsplätzen, an denen ein Produkt schon vollständig bearbeitet worden sein muss,
bevor es an dem aktuellen Arbeitsplatz bearbeitet werden kann. % besteht keine Vorgängerbeziehung zwischen zwei Arbeitsplätzen,
% kann also gleichzeitig am Produkt gearbeitet werden (wenn z.B. ein Arbeitsplatz der vorherigen Station über diese hinaus arbeitet)

Des Weiteren können für jeden Arbeitsplatz erlaubte \emph{Drifts} festgelegt werden:
Die Arbeit an diesem Platz darf dann um diese Zeiteinheit früher beginnen bzw.\
über das Ende der Taktzeit hinausdauern.
Bei der ersten Arbeitsstation kann es zudem zu einem initialen Drift kommen,
der z.B.\ durch Verspätungen aus einer vorangegangenen Schicht (mit eigener Reihenfolgeplanung) entstanden ist.
Die Driftgrenzen dürfen nie überschritten werden,
da das Produkt dann beispielsweise außerhalb des Arbeitsradius bestimmter Werkzeuge gelangt und das Fließband gestoppt werden müsste.
Wenn der maximale Drift an einem Arbeitsplatz überschritten würde, muss daher eine sogenannte \emph{Springerin} eingesetzt werden,
die dann pünktlich den Folgeauftrag übernimmt, während die Person an dem Arbeitsplatz ihren aktuellen Auftrag beendet.
Diese Person kann dann den Folgeauftrag, der ja bereits von der Springerin bearbeitet wird,
überspringen und pünktlich zum Taktschlag mit dem nächsten Auftrag fortfahren.

\paragraph{Die Aufträge} bestehen aus einer Auflistung verschiedener Arbeitsschritte, die auf den Arbeitsstationen ausgeführt werden müssen,
\marginpar{\scriptsize {\emph{Auftrag}:\\ \phantom{s} Zeit \\ \phantom{s} Attribute}}
um ein konkretes Produkt herzustellen.
Pro Auftrag kennen wir die benötigte \emph{Arbeitszeit} pro Arbeitsplatz und eine Liste von \emph{Attributen},
anhand derer Regeln für die Reihenfolge bestimmt werden können.
\todo{\tiny Sollen Attribute zunächst vernachlässigt werden? Wenn nicht, sollte das Beispiel ggf. erweitert werden}
Ein vereinfachtes Beispiel ohne Attribute ist in~\cref{fig:sequencing_auftraege} dargestellt.

\paragraph{Die Regeln} schränken den Lösungsraum, also die Anzahl zulässiger Reihenfolgen für die Aufträge,
zusätzlich ein.
So lässt sich beispielsweise festlegen, dass Aufträge mit einem bestimmten Attribut einen gewissen zeitlichen Mindestabstand haben.
Die Regeln stellen jedoch keine starren Vorschriften dar, sondern sollen so gut es geht erfüllt werden.
% ,weshalb sich ihre Nichterfüllung negativ auf den Zielfunktionswert auswirkt.

\paragraph{Die Lösung} einer konkreten Probleminstanz beinhaltet eine hinsichtlich der Zielfunktion optimierte Reihenfolge der Aufträge.
Konkret setzt sich der Zielfunktionswert aus drei Komponenten zusammen, die je nach Wichtigkeit mit einem festgelegten Faktor in die Zielfunktion eingehen:
Die Anzahl an benötigten Springerinnen, der Drift und die Anzahl an verletzen Regeln.
Im einfachen Beispiel (ohne Attribute und Regeln) in~\cref{fig:example_jumper} würden dementsprechend die benötigte Springerin
und die Drifts an Station 1 und 2 in die Zielfunktion eingehen.
\cref{fig:example_optimal} stellt zudem eine Lösung der Instanz dar, die ohne Springerinnen auskommt.

\begin{figure}[H]
    \centering
    \begin{subfigure}[t]{0.48\textwidth}
        \resizebox{1\textwidth}{!}{
            \tikzfig{images/reihenfolgeplanung/example_jumper}
        }
        \caption{%
        \label{fig:example_jumper}
        Die hier angegebene (nicht optimale) Lösung erfordert eine Springerin (rot),
        da der dritte Auftrag sonst den zulässigen maximalen Drift an Station 1 überschritte.
        }
    \end{subfigure}%
    \hfill
    \begin{subfigure}[t]{0.48\textwidth}
        \resizebox{1\textwidth}{!}{
            \tikzfig{images/reihenfolgeplanung/example_optimal}
        }
        \caption{%
        \label{fig:example_optimal}
        Diese Lösung ist hinsichtlich der Anzahl der Springerinnen und der aufsummierten Drifts optimal.
        }
    \end{subfigure}
    \caption{%
    \label{fig:example_sequencing}
    Beispiel für zwei Lösungen der Instanz aus \cref{fig:sequencing_instance}.
    Die festgelegte Auftragsreihenfolge liest sich von oben nach unten
    und die Bearbeitungszeiten der Aufträge sind durch die Länge der horizontalen Balken symbolisiert.
    Zudem markieren die gestrichelten Linien die maximalen Drifts an den Stationen.
    Da die Arbeit an Station 2 erst nach dem Arbeitsende an Station 1 beginnen kann,
    kann ein Drift an Station 1 den Bearbeitungsstart an Station 2 verzögern.
    Der Startzeitpunkt auf Station 2 ergibt sich also aus dem Maximum von dem Ende des Auftrags auf Station 1
    und dem Drift des vorherigen Auftrags auf Station 2.
    Station 3 hingegen hat keine Vorgängerbeziehungen und kann alle Aufträge pünktlich starten,
    weil sich ein Drift auf Station 2 nicht auf Station 3 auswirkt,
    sondern die Arbeiten an diesen Stationen gleichzeitig erledigt werden können.}
\end{figure}

% Input Parameter:
%Eine Probleminstanz für die Reihenfolgeplanung enthält folgende Informationen:
%\begin{itemize}
%    \item Für das Fließband (\textit{line}) gibt es:
%    \begin{itemize}
%        \item eine Liste an (verschiedenen) Arbeitsstationen (\textit{stations}) mit:
%        \begin{itemize}
%            \item Anzahl Takte (\textit{cycles}) %ermöglicht unterschiedliche Taktzeiten für die Stationen trotz globaler cycle time
%            \item einer Liste an (verschiedenen) Arbeitsplätzen (\textit{workplaces}) mit:
%            \begin{itemize}
%                \item Anzahl Takte (\textit{cycles})
%                \item Liste an Vorgängern (\textit{predecessors})
%                \item Informationen über den Drift (\textit{maxEarlyDrift, maxLateDrift, initialDrift})
%            \end{itemize}
%        \end{itemize}
%        \item eine globale Taktzeit (\textit{cycle time}) pro Arbeitsstation (wobei eine Station jedoch mehrere Takte dauern kann)
%    \end{itemize}
%    \item Außerdem gibt es eine Anzahl an Aufträgen (\textit{numOrders}), welche aus verschiedenen Arbeitsschritten,
%    die bei der Fließbandabstimmung (siehe \cref{subsec:anwendungsfall-fliessbandabstimmung}) beachtet wurden und
%    hier nicht mehr relevant sind.
%    \item Für jeden Auftrag (\textit{order}) kennen wir:
%    \begin{itemize}
%        \item die benötigte Arbeitszeit (\textit{processingTime}) an einer gegebenen Arbeitsstation
%        \item eine Liste an Attributen (\textit{Attributes})  % kein input -> automatisch erstellt
%    \end{itemize}
%    \item Zudem gibt es diverse Anforderungen an bzw.\ Regeln für die Reihenfolge
    %https://gitlab.optware.de/sequencer/product/90200438-sequencer-mvp/-/blob/main/libs/interfaces/src/assets/problem/problem_data_schema.json?ref_type=heads
%\end{itemize}

% Output -> result.svg
%Eine Lösung, also eine konkrete Reihenfolgeplanung, enthält
%\begin{itemize}
%    \item eine Reihenfolge der Aufträge (\textit{SequenceResult})
%    \item und einen Zielfunktionswert, der sich zusammensetzt aus:
%    \begin{itemize}
%        \item jumper
%        \item sequencingRules % Anzahl verletzter Regeln
%        \item lateDrift
%    \end{itemize}
%\end{itemize}

\subsubsection*{Datensatz}\label{subsubsec:datensatz-sequencer}
% https://gitlab.optware.de/sequencer/product/90200438-sequencer-mvp/-/blob/main/libs/interfaces/src/assets/data-generation/generation_input_scheme.json?ref_type=heads
Der Datensatz wurde mittels eines \textit{data generators} künstlich erzeugt.
Dafür wurden für die verschiedenen Parameter Werte ausgewählt bzw.\ ein Bereich angegeben, aus dem der
Wert des Parameters gleichverteilt ausgewählt wurde, um Instanzen verschiedener realer Probleme zu simulieren.

Um die Komplexität des Problems zu verringern, wurden in Bezug auf das oben präsentierte Problem folgende
Vereinfachungen vorgenommen:
\todo[inline]{Liste von Vereinfachungen updaten!}
\begin{itemize}
    \item Jede Station besteht aus nur einem Arbeitsplatz,
    d.h.\ die Aufteilung einer Arbeitsstation in verschiedene Arbeitsplätze wurde weggelassen.
    \item \dots
\end{itemize}
\todo[inline]{Nachdem die Parameter für den Datengenerator festgelegt wurden, sollte hier angegeben werden,
    wie die Parameter gesetzt wurden und wie viele Instanzen welcher Größe generiert wurden.
    Vielleicht kann der generierte Datensatz mit Grafiken/Tabellen dargestellt werden (wie viele Instanzen welcher Art, etc.)?!}
% für folgende Werte kann entweder eine Zahl oder eine Verteilung (number-or-distribution) angegeben werden:
    % cycles
    % numPredecessors
    % numWorkplaces
    % numStations
    % workloadDistribution (pro StationId -> aus dieser Verteilung wird gesamplet, um die processing time pro station zu ermitteln)
% für folgende Werte kann entweder eine Zeiteinheit oder eine Verteilung (time-unit-or-distribution) angegeben werden:
    % maxLateDrift
        % sind keine Drift Parameter angegeben, dann ist an dieser Workstation kein Drift erlaubt
    % cycleTime
%TODO: die Attribute werden automatisch erstellt. An welcher Stelle passiert das? Und wie?
\subsubsection{Anwendungsfall Reihenfolgeplanung (Sequencing)}\label{subsubsec:anwendungsfall-reihenfolgeplanung}
% Allgemeine (high-level) Beschreibung:
Wie bereits in~\cref{subsec:fliessbandfertigung} beschrieben,
folgt in modernen Produktionen auf die Fließbandabstimmung,
in der die benötigten Arbeitsschritte auf Arbeitsstationen verteilt wurden,
häufig die operative Reihenfolge(fein)planung, auch \emph{Sequencing} genannt.
Hierbei geht es nun darum, konkrete Arbeitsaufträge,
die die einzelnen Stationen durchlaufen müssen, in die optimale Reihenfolge zu bringen.
Wir legen hier eine sogenannte \emph{mixed-model assembly line} zugrunde,
was bedeutet, dass die Aufträge die Herstellung verschiedener Modelle eines Basisprodukts beschreiben.
Die benötigten Arbeitsschritte und die Zeit für deren Erledigung unterscheiden sich somit pro Auftrag.
Die operative Feinplanung muss daher u.a.\ dafür sorgen, dass zu keinem Zeitpunkt zu große Abweichungen (Drift)
von den in der Fließbandabstimmung verwendeten Mittelwerten pro Station entstehen.

Eine Probleminstanz für das Sequencing besteht aus
einem Fließband, einer Liste an Aufträgen und einer Liste an Anforderungen bzw.\ Regeln für die Reihenfolge.
Das Ziel ist es, die Aufträge in eine Reihenfolge zu bringen, die diese Regeln beachtet und eine vorher definierte Zielfunktion optimiert.
Im Folgenden werden nun die Eingabeparameter und die verschiedenen Komponenten der Zielfunktion und einer validen Lösung genauer erläutert.

\begin{figure}[H]
    \begin{subfigure}[c]{0.48\textwidth}
        \centering
        \tikzfig{images/reihenfolgeplanung/assembly_line}
        \caption{%
        \label{fig:sequencing_assembly_line}
        Ein Fließband mit drei Stationen,
            die jeweils aus einem Arbeitsplatz bestehen,
            einen Takt dauern
            und eine maximale Driftzeit erlauben.
        Ein Auftrag an Station 1 muss vollständig abgearbeitet sein,
            bevor er an Station 2 weiterbearbeitet werden kann (durchgezogene Linie).
        }
    \end{subfigure}%
    \hfill
    \begin{subfigure}[c]{0.48\textwidth}
        \centering
        \tikzfig{images/reihenfolgeplanung/auftraege}
        \caption{%
        \label{fig:sequencing_auftraege}
        Die Aufträge symbolisieren durch die verschiedenen Formen
        die Herstellung verschiedener Varianten eines Basisprodukts
        und haben eine individuelle Bearbeitungszeit pro Station.
        Die zu erledigenden Aufträge sind vorgegeben.
        }
    \end{subfigure}
    \caption{%
        \label{fig:sequencing_instance}
        Eine einfache Probleminstanz für das Sequencing.
        Das Ziel ist eine optimale Abarbeitungsreihenfolge für die Auftragsliste zu finden.
    }
\end{figure}

\paragraph{Das Fließband}
ist üblicherweise bereits durch die Fließbandabstimmung (siehe~\cref{subsec:anwendungsfall-fliessbandabstimmung})
\marginpar{\scriptsize {\emph{Fließband}:\\ \phantom{s} Taktzeit \\ \phantom{s} Stationen}}
konfiguriert worden und hat somit eine feste Anzahl an \emph{Arbeitsstationen}---denen die verschiedenen Arbeitsschritte,
die an einem Produkt ausgeführt werden müssen, zugeteilt wurden---und eine feste \emph{Taktzeit}.
Ein einfaches Beispiel für ein Fließband mit drei Stationen ist in~\cref{fig:sequencing_assembly_line} illustriert.

Jede Station hat eine vordefinierte Anzahl an \emph{Takten}, für die ein Produkt auf dieser Station verweilen kann,
\marginpar{\scriptsize {\emph{Station}:\\ \phantom{s} Takte \\ \phantom{s} Plätze}}
und ist selbst wieder in eine Menge an \emph{Arbeitsplätze} aufgeteilt.
Jeder Arbeitsplatz kann das Produkt selbst wieder für eine feste Anzahl an Takten bearbeiten
(was auch dazu führen kann, dass sich ein Arbeitsplatz über mehrere Stationen erstreckt)
\marginpar{\scriptsize {\emph{Arbeitsplatz}:\\ \phantom{s} Takte \\ \phantom{s} Vorgänger \\ \phantom{s} Drifts}}
und hat eine Liste an Vorgängerarbeitsplätzen, an denen ein Produkt schon vollständig bearbeitet worden sein muss,
bevor es an dem aktuellen Arbeitsplatz bearbeitet werden kann. % besteht keine Vorgängerbeziehung zwischen zwei Arbeitsplätzen,
% kann also gleichzeitig am Produkt gearbeitet werden (wenn z.B. ein Arbeitsplatz der vorherigen Station über diese hinaus arbeitet)

Des Weiteren können für jeden Arbeitsplatz erlaubte \emph{Drifts} festgelegt werden:
Die Arbeit an diesem Platz darf dann um diese Zeiteinheit früher beginnen bzw.\
über das Ende der Taktzeit hinausdauern.
Bei der ersten Arbeitsstation kann es zudem zu einem initialen Drift kommen,
der z.B.\ durch Verspätungen aus einer vorangegangenen Schicht (mit eigener Reihenfolgeplanung) entstanden ist.
Die Driftgrenzen dürfen nie überschritten werden,
da das Produkt dann beispielsweise außerhalb des Arbeitsradius bestimmter Werkzeuge gelangt und das Fließband gestoppt werden müsste.
Wenn der maximale Drift an einem Arbeitsplatz überschritten würde, muss daher eine sogenannte \emph{Springerin} eingesetzt werden,
die dann pünktlich den Folgeauftrag übernimmt, während die Person an dem Arbeitsplatz ihren aktuellen Auftrag beendet.
Diese Person kann dann den Folgeauftrag, der ja bereits von der Springerin bearbeitet wird,
überspringen und pünktlich zum Taktschlag mit dem nächsten Auftrag fortfahren.

\paragraph{Die Aufträge} bestehen aus einer Auflistung verschiedener Arbeitsschritte, die auf den Arbeitsstationen ausgeführt werden müssen,
\marginpar{\scriptsize {\emph{Auftrag}:\\ \phantom{s} Zeit \\ \phantom{s} Attribute}}
um ein konkretes Produkt herzustellen.
Pro Auftrag kennen wir die benötigte \emph{Arbeitszeit} pro Arbeitsplatz und eine Liste von \emph{Attributen},
anhand derer Regeln für die Reihenfolge bestimmt werden können.
\todo{\tiny Sollen Attribute zunächst vernachlässigt werden? Wenn nicht, sollte das Beispiel ggf. erweitert werden}
Ein vereinfachtes Beispiel ohne Attribute ist in~\cref{fig:sequencing_auftraege} dargestellt.

\paragraph{Die Regeln} schränken den Lösungsraum, also die Anzahl zulässiger Reihenfolgen für die Aufträge,
zusätzlich ein.
So lässt sich beispielsweise festlegen, dass Aufträge mit einem bestimmten Attribut einen gewissen zeitlichen Mindestabstand haben.
Die Regeln stellen jedoch keine starren Vorschriften dar, sondern sollen so gut es geht erfüllt werden.
% ,weshalb sich ihre Nichterfüllung negativ auf den Zielfunktionswert auswirkt.

\paragraph{Die Lösung} einer konkreten Probleminstanz beinhaltet eine hinsichtlich der Zielfunktion optimierte Reihenfolge der Aufträge.
Konkret setzt sich der Zielfunktionswert aus drei Komponenten zusammen, die je nach Wichtigkeit mit einem festgelegten Faktor in die Zielfunktion eingehen:
Die Anzahl an benötigten Springerinnen, der Drift und die Anzahl an verletzen Regeln.
Im einfachen Beispiel (ohne Attribute und Regeln) in~\cref{fig:example_jumper} würden dementsprechend die benötigte Springerin
und die Drifts an Station 1 und 2 in die Zielfunktion eingehen.
\cref{fig:example_optimal} stellt zudem eine Lösung der Instanz dar, die ohne Springerinnen auskommt.

\begin{figure}[H]
    \centering
    \begin{subfigure}[t]{0.48\textwidth}
        \resizebox{1\textwidth}{!}{
            \tikzfig{images/reihenfolgeplanung/example_jumper}
        }
        \caption{%
        \label{fig:example_jumper}
        Die hier angegebene (nicht optimale) Lösung erfordert eine Springerin (rot),
        da der dritte Auftrag sonst den zulässigen maximalen Drift an Station 1 überschritte.
        }
    \end{subfigure}%
    \hfill
    \begin{subfigure}[t]{0.48\textwidth}
        \resizebox{1\textwidth}{!}{
            \tikzfig{images/reihenfolgeplanung/example_optimal}
        }
        \caption{%
        \label{fig:example_optimal}
        Diese Lösung ist hinsichtlich der Anzahl der Springerinnen und der aufsummierten Drifts optimal.
        }
    \end{subfigure}
    \caption{%
    \label{fig:example_sequencing}
    Beispiel für zwei Lösungen der Instanz aus \cref{fig:sequencing_instance}.
    Die festgelegte Auftragsreihenfolge liest sich von oben nach unten
    und die Bearbeitungszeiten der Aufträge sind durch die Länge der horizontalen Balken symbolisiert.
    Zudem markieren die gestrichelten Linien die maximalen Drifts an den Stationen.
    Da die Arbeit an Station 2 erst nach dem Arbeitsende an Station 1 beginnen kann,
    kann ein Drift an Station 1 den Bearbeitungsstart an Station 2 verzögern.
    Der Startzeitpunkt auf Station 2 ergibt sich also aus dem Maximum von dem Ende des Auftrags auf Station 1
    und dem Drift des vorherigen Auftrags auf Station 2.
    Station 3 hingegen hat keine Vorgängerbeziehungen und kann alle Aufträge pünktlich starten,
    weil sich ein Drift auf Station 2 nicht auf Station 3 auswirkt,
    sondern die Arbeiten an diesen Stationen gleichzeitig erledigt werden können.}
\end{figure}

% Input Parameter:
%Eine Probleminstanz für die Reihenfolgeplanung enthält folgende Informationen:
%\begin{itemize}
%    \item Für das Fließband (\textit{line}) gibt es:
%    \begin{itemize}
%        \item eine Liste an (verschiedenen) Arbeitsstationen (\textit{stations}) mit:
%        \begin{itemize}
%            \item Anzahl Takte (\textit{cycles}) %ermöglicht unterschiedliche Taktzeiten für die Stationen trotz globaler cycle time
%            \item einer Liste an (verschiedenen) Arbeitsplätzen (\textit{workplaces}) mit:
%            \begin{itemize}
%                \item Anzahl Takte (\textit{cycles})
%                \item Liste an Vorgängern (\textit{predecessors})
%                \item Informationen über den Drift (\textit{maxEarlyDrift, maxLateDrift, initialDrift})
%            \end{itemize}
%        \end{itemize}
%        \item eine globale Taktzeit (\textit{cycle time}) pro Arbeitsstation (wobei eine Station jedoch mehrere Takte dauern kann)
%    \end{itemize}
%    \item Außerdem gibt es eine Anzahl an Aufträgen (\textit{numOrders}), welche aus verschiedenen Arbeitsschritten,
%    die bei der Fließbandabstimmung (siehe \cref{subsec:anwendungsfall-fliessbandabstimmung}) beachtet wurden und
%    hier nicht mehr relevant sind.
%    \item Für jeden Auftrag (\textit{order}) kennen wir:
%    \begin{itemize}
%        \item die benötigte Arbeitszeit (\textit{processingTime}) an einer gegebenen Arbeitsstation
%        \item eine Liste an Attributen (\textit{Attributes})  % kein input -> automatisch erstellt
%    \end{itemize}
%    \item Zudem gibt es diverse Anforderungen an bzw.\ Regeln für die Reihenfolge
    %https://gitlab.optware.de/sequencer/product/90200438-sequencer-mvp/-/blob/main/libs/interfaces/src/assets/problem/problem_data_schema.json?ref_type=heads
%\end{itemize}

% Output -> result.svg
%Eine Lösung, also eine konkrete Reihenfolgeplanung, enthält
%\begin{itemize}
%    \item eine Reihenfolge der Aufträge (\textit{SequenceResult})
%    \item und einen Zielfunktionswert, der sich zusammensetzt aus:
%    \begin{itemize}
%        \item jumper
%        \item sequencingRules % Anzahl verletzter Regeln
%        \item lateDrift
%    \end{itemize}
%\end{itemize}

\subsubsection*{Datensatz}\label{subsubsec:datensatz-sequencer}
% https://gitlab.optware.de/sequencer/product/90200438-sequencer-mvp/-/blob/main/libs/interfaces/src/assets/data-generation/generation_input_scheme.json?ref_type=heads
Der Datensatz wurde mittels eines \textit{data generators} künstlich erzeugt.
Dafür wurden für die verschiedenen Parameter Werte ausgewählt bzw.\ ein Bereich angegeben, aus dem der
Wert des Parameters gleichverteilt ausgewählt wurde, um Instanzen verschiedener realer Probleme zu simulieren.

Um die Komplexität des Problems zu verringern, wurden in Bezug auf das oben präsentierte Problem folgende
Vereinfachungen vorgenommen:
\todo[inline]{Liste von Vereinfachungen updaten!}
\begin{itemize}
    \item Jede Station besteht aus nur einem Arbeitsplatz,
    d.h.\ die Aufteilung einer Arbeitsstation in verschiedene Arbeitsplätze wurde weggelassen.
    \item \dots
\end{itemize}
\todo[inline]{Nachdem die Parameter für den Datengenerator festgelegt wurden, sollte hier angegeben werden,
    wie die Parameter gesetzt wurden und wie viele Instanzen welcher Größe generiert wurden.
    Vielleicht kann der generierte Datensatz mit Grafiken/Tabellen dargestellt werden (wie viele Instanzen welcher Art, etc.)?!}
% für folgende Werte kann entweder eine Zahl oder eine Verteilung (number-or-distribution) angegeben werden:
    % cycles
    % numPredecessors
    % numWorkplaces
    % numStations
    % workloadDistribution (pro StationId -> aus dieser Verteilung wird gesamplet, um die processing time pro station zu ermitteln)
% für folgende Werte kann entweder eine Zeiteinheit oder eine Verteilung (time-unit-or-distribution) angegeben werden:
    % maxLateDrift
        % sind keine Drift Parameter angegeben, dann ist an dieser Workstation kein Drift erlaubt
    % cycleTime
%TODO: die Attribute werden automatisch erstellt. An welcher Stelle passiert das? Und wie?

    \section{Anwendungsfälle}

Im Folgenden sollen zunächst die industriellen Anwendungsfälle vorgestellt werden.

\subsection{Fließbandfertigung}\label{subsec:fliessbandfertigung}

Moderne Produktion, wie zum Beispiel die Automobilproduktion, erfolgt oft mithilfe der Fließbandfertigung.
Bei der Fließbandfertigung durchläuft das zu fertigende Produkt auf dem Weg zur Fertigstellung zahlreiche
Arbeitsstationen, in denen die einzelnen Arbeitsschritte zur Fertigstellung des Produktes ausgeführt werden.

Die Beplanung einer Fertigungslinie erfolgt dabei in zwei Schritten:
Die Fließbandabstimmung (cf.\ \cref{subsec:anwendungsfall-fliessbandabstimmung}) und die Reihenfolgeplanung
(cf.\ \cref{subsubsec:anwendungsfall-reihenfolgeplanung}).

Die Fließbandabstimmung (\emph{Assembly Line Balancing}) beschäftigt sich damit,
Arbeitsschritte nach ihrem mittleren Zeitbedarf auf Arbeitsstationen zu verteilen.
In der Reihenfolgeplanung (\emph{Sequencing}) werden konkrete Folgen von Aufträgen gebildet,
die in Form einer operativen Feinplanung zu große Abweichung von den in der
Fließbandabstimmung verwendeten Mittelwerten auf den einzelnen Stationen verhindern sollen.
Die Fließbandabstimmung liefert somit eine taktische Grobplanung,
auf der die operative Reihenfolge(fein)planung basiert.

Der mittlere Zeitbedarf der Arbeitsschritte, die als Ausgangsbasis der Fließbandabstimmung dient,
ergibt sich aus dem tatsächlichen Zeitbedarf für einen Arbeitsschritt multipliziert
mit der erwarteten Häufigkeit des Auftretens dieses Schrittes.
Dauert z.B.\ die Montage einer Anhängerkupplung 100 Sekunden,
und ist eine Anhängerkupplung nur bei jedem zehnten Fahrzeug zu erwarten,
beträgt der mittlere Zeitbedarf 10 Sekunden.

\subsubsection{Anwendungsfall Reihenfolgeplanung (Sequencing)}\label{subsubsec:anwendungsfall-reihenfolgeplanung}
% Allgemeine (high-level) Beschreibung:
Wie bereits in~\cref{subsec:fliessbandfertigung} beschrieben,
folgt in modernen Produktionen auf die Fließbandabstimmung,
in der die benötigten Arbeitsschritte auf Arbeitsstationen verteilt wurden,
häufig die operative Reihenfolge(fein)planung, auch \emph{Sequencing} genannt.
Hierbei geht es nun darum, konkrete Arbeitsaufträge,
die die einzelnen Stationen durchlaufen müssen, in die optimale Reihenfolge zu bringen.
Wir legen hier eine sogenannte \emph{mixed-model assembly line} zugrunde,
was bedeutet, dass die Aufträge die Herstellung verschiedener Modelle eines Basisprodukts beschreiben.
Die benötigten Arbeitsschritte und die Zeit für deren Erledigung unterscheiden sich somit pro Auftrag.
Die operative Feinplanung muss daher u.a.\ dafür sorgen, dass zu keinem Zeitpunkt zu große Abweichungen (Drift)
von den in der Fließbandabstimmung verwendeten Mittelwerten pro Station entstehen.

Eine Probleminstanz für das Sequencing besteht aus
einem Fließband, einer Liste an Aufträgen und einer Liste an Anforderungen bzw.\ Regeln für die Reihenfolge.
Das Ziel ist es, die Aufträge in eine Reihenfolge zu bringen, die diese Regeln beachtet und eine vorher definierte Zielfunktion optimiert.
Im Folgenden werden nun die Eingabeparameter und die verschiedenen Komponenten der Zielfunktion und einer validen Lösung genauer erläutert.

\begin{figure}[H]
    \begin{subfigure}[c]{0.48\textwidth}
        \centering
        \tikzfig{images/reihenfolgeplanung/assembly_line}
        \caption{%
        \label{fig:sequencing_assembly_line}
        Ein Fließband mit drei Stationen,
            die jeweils aus einem Arbeitsplatz bestehen,
            einen Takt dauern
            und eine maximale Driftzeit erlauben.
        Ein Auftrag an Station 1 muss vollständig abgearbeitet sein,
            bevor er an Station 2 weiterbearbeitet werden kann (durchgezogene Linie).
        }
    \end{subfigure}%
    \hfill
    \begin{subfigure}[c]{0.48\textwidth}
        \centering
        \tikzfig{images/reihenfolgeplanung/auftraege}
        \caption{%
        \label{fig:sequencing_auftraege}
        Die Aufträge symbolisieren durch die verschiedenen Formen
        die Herstellung verschiedener Varianten eines Basisprodukts
        und haben eine individuelle Bearbeitungszeit pro Station.
        Die zu erledigenden Aufträge sind vorgegeben.
        }
    \end{subfigure}
    \caption{%
        \label{fig:sequencing_instance}
        Eine einfache Probleminstanz für das Sequencing.
        Das Ziel ist eine optimale Abarbeitungsreihenfolge für die Auftragsliste zu finden.
    }
\end{figure}

\paragraph{Das Fließband}
ist üblicherweise bereits durch die Fließbandabstimmung (siehe~\cref{subsec:anwendungsfall-fliessbandabstimmung})
\marginpar{\scriptsize {\emph{Fließband}:\\ \phantom{s} Taktzeit \\ \phantom{s} Stationen}}
konfiguriert worden und hat somit eine feste Anzahl an \emph{Arbeitsstationen}---denen die verschiedenen Arbeitsschritte,
die an einem Produkt ausgeführt werden müssen, zugeteilt wurden---und eine feste \emph{Taktzeit}.
Ein einfaches Beispiel für ein Fließband mit drei Stationen ist in~\cref{fig:sequencing_assembly_line} illustriert.

Jede Station hat eine vordefinierte Anzahl an \emph{Takten}, für die ein Produkt auf dieser Station verweilen kann,
\marginpar{\scriptsize {\emph{Station}:\\ \phantom{s} Takte \\ \phantom{s} Plätze}}
und ist selbst wieder in eine Menge an \emph{Arbeitsplätze} aufgeteilt.
Jeder Arbeitsplatz kann das Produkt selbst wieder für eine feste Anzahl an Takten bearbeiten
(was auch dazu führen kann, dass sich ein Arbeitsplatz über mehrere Stationen erstreckt)
\marginpar{\scriptsize {\emph{Arbeitsplatz}:\\ \phantom{s} Takte \\ \phantom{s} Vorgänger \\ \phantom{s} Drifts}}
und hat eine Liste an Vorgängerarbeitsplätzen, an denen ein Produkt schon vollständig bearbeitet worden sein muss,
bevor es an dem aktuellen Arbeitsplatz bearbeitet werden kann. % besteht keine Vorgängerbeziehung zwischen zwei Arbeitsplätzen,
% kann also gleichzeitig am Produkt gearbeitet werden (wenn z.B. ein Arbeitsplatz der vorherigen Station über diese hinaus arbeitet)

Des Weiteren können für jeden Arbeitsplatz erlaubte \emph{Drifts} festgelegt werden:
Die Arbeit an diesem Platz darf dann um diese Zeiteinheit früher beginnen bzw.\
über das Ende der Taktzeit hinausdauern.
Bei der ersten Arbeitsstation kann es zudem zu einem initialen Drift kommen,
der z.B.\ durch Verspätungen aus einer vorangegangenen Schicht (mit eigener Reihenfolgeplanung) entstanden ist.
Die Driftgrenzen dürfen nie überschritten werden,
da das Produkt dann beispielsweise außerhalb des Arbeitsradius bestimmter Werkzeuge gelangt und das Fließband gestoppt werden müsste.
Wenn der maximale Drift an einem Arbeitsplatz überschritten würde, muss daher eine sogenannte \emph{Springerin} eingesetzt werden,
die dann pünktlich den Folgeauftrag übernimmt, während die Person an dem Arbeitsplatz ihren aktuellen Auftrag beendet.
Diese Person kann dann den Folgeauftrag, der ja bereits von der Springerin bearbeitet wird,
überspringen und pünktlich zum Taktschlag mit dem nächsten Auftrag fortfahren.

\paragraph{Die Aufträge} bestehen aus einer Auflistung verschiedener Arbeitsschritte, die auf den Arbeitsstationen ausgeführt werden müssen,
\marginpar{\scriptsize {\emph{Auftrag}:\\ \phantom{s} Zeit \\ \phantom{s} Attribute}}
um ein konkretes Produkt herzustellen.
Pro Auftrag kennen wir die benötigte \emph{Arbeitszeit} pro Arbeitsplatz und eine Liste von \emph{Attributen},
anhand derer Regeln für die Reihenfolge bestimmt werden können.
\todo{\tiny Sollen Attribute zunächst vernachlässigt werden? Wenn nicht, sollte das Beispiel ggf. erweitert werden}
Ein vereinfachtes Beispiel ohne Attribute ist in~\cref{fig:sequencing_auftraege} dargestellt.

\paragraph{Die Regeln} schränken den Lösungsraum, also die Anzahl zulässiger Reihenfolgen für die Aufträge,
zusätzlich ein.
So lässt sich beispielsweise festlegen, dass Aufträge mit einem bestimmten Attribut einen gewissen zeitlichen Mindestabstand haben.
Die Regeln stellen jedoch keine starren Vorschriften dar, sondern sollen so gut es geht erfüllt werden.
% ,weshalb sich ihre Nichterfüllung negativ auf den Zielfunktionswert auswirkt.

\paragraph{Die Lösung} einer konkreten Probleminstanz beinhaltet eine hinsichtlich der Zielfunktion optimierte Reihenfolge der Aufträge.
Konkret setzt sich der Zielfunktionswert aus drei Komponenten zusammen, die je nach Wichtigkeit mit einem festgelegten Faktor in die Zielfunktion eingehen:
Die Anzahl an benötigten Springerinnen, der Drift und die Anzahl an verletzen Regeln.
Im einfachen Beispiel (ohne Attribute und Regeln) in~\cref{fig:example_jumper} würden dementsprechend die benötigte Springerin
und die Drifts an Station 1 und 2 in die Zielfunktion eingehen.
\cref{fig:example_optimal} stellt zudem eine Lösung der Instanz dar, die ohne Springerinnen auskommt.

\begin{figure}[H]
    \centering
    \begin{subfigure}[t]{0.48\textwidth}
        \resizebox{1\textwidth}{!}{
            \tikzfig{images/reihenfolgeplanung/example_jumper}
        }
        \caption{%
        \label{fig:example_jumper}
        Die hier angegebene (nicht optimale) Lösung erfordert eine Springerin (rot),
        da der dritte Auftrag sonst den zulässigen maximalen Drift an Station 1 überschritte.
        }
    \end{subfigure}%
    \hfill
    \begin{subfigure}[t]{0.48\textwidth}
        \resizebox{1\textwidth}{!}{
            \tikzfig{images/reihenfolgeplanung/example_optimal}
        }
        \caption{%
        \label{fig:example_optimal}
        Diese Lösung ist hinsichtlich der Anzahl der Springerinnen und der aufsummierten Drifts optimal.
        }
    \end{subfigure}
    \caption{%
    \label{fig:example_sequencing}
    Beispiel für zwei Lösungen der Instanz aus \cref{fig:sequencing_instance}.
    Die festgelegte Auftragsreihenfolge liest sich von oben nach unten
    und die Bearbeitungszeiten der Aufträge sind durch die Länge der horizontalen Balken symbolisiert.
    Zudem markieren die gestrichelten Linien die maximalen Drifts an den Stationen.
    Da die Arbeit an Station 2 erst nach dem Arbeitsende an Station 1 beginnen kann,
    kann ein Drift an Station 1 den Bearbeitungsstart an Station 2 verzögern.
    Der Startzeitpunkt auf Station 2 ergibt sich also aus dem Maximum von dem Ende des Auftrags auf Station 1
    und dem Drift des vorherigen Auftrags auf Station 2.
    Station 3 hingegen hat keine Vorgängerbeziehungen und kann alle Aufträge pünktlich starten,
    weil sich ein Drift auf Station 2 nicht auf Station 3 auswirkt,
    sondern die Arbeiten an diesen Stationen gleichzeitig erledigt werden können.}
\end{figure}

% Input Parameter:
%Eine Probleminstanz für die Reihenfolgeplanung enthält folgende Informationen:
%\begin{itemize}
%    \item Für das Fließband (\textit{line}) gibt es:
%    \begin{itemize}
%        \item eine Liste an (verschiedenen) Arbeitsstationen (\textit{stations}) mit:
%        \begin{itemize}
%            \item Anzahl Takte (\textit{cycles}) %ermöglicht unterschiedliche Taktzeiten für die Stationen trotz globaler cycle time
%            \item einer Liste an (verschiedenen) Arbeitsplätzen (\textit{workplaces}) mit:
%            \begin{itemize}
%                \item Anzahl Takte (\textit{cycles})
%                \item Liste an Vorgängern (\textit{predecessors})
%                \item Informationen über den Drift (\textit{maxEarlyDrift, maxLateDrift, initialDrift})
%            \end{itemize}
%        \end{itemize}
%        \item eine globale Taktzeit (\textit{cycle time}) pro Arbeitsstation (wobei eine Station jedoch mehrere Takte dauern kann)
%    \end{itemize}
%    \item Außerdem gibt es eine Anzahl an Aufträgen (\textit{numOrders}), welche aus verschiedenen Arbeitsschritten,
%    die bei der Fließbandabstimmung (siehe \cref{subsec:anwendungsfall-fliessbandabstimmung}) beachtet wurden und
%    hier nicht mehr relevant sind.
%    \item Für jeden Auftrag (\textit{order}) kennen wir:
%    \begin{itemize}
%        \item die benötigte Arbeitszeit (\textit{processingTime}) an einer gegebenen Arbeitsstation
%        \item eine Liste an Attributen (\textit{Attributes})  % kein input -> automatisch erstellt
%    \end{itemize}
%    \item Zudem gibt es diverse Anforderungen an bzw.\ Regeln für die Reihenfolge
    %https://gitlab.optware.de/sequencer/product/90200438-sequencer-mvp/-/blob/main/libs/interfaces/src/assets/problem/problem_data_schema.json?ref_type=heads
%\end{itemize}

% Output -> result.svg
%Eine Lösung, also eine konkrete Reihenfolgeplanung, enthält
%\begin{itemize}
%    \item eine Reihenfolge der Aufträge (\textit{SequenceResult})
%    \item und einen Zielfunktionswert, der sich zusammensetzt aus:
%    \begin{itemize}
%        \item jumper
%        \item sequencingRules % Anzahl verletzter Regeln
%        \item lateDrift
%    \end{itemize}
%\end{itemize}

\subsubsection*{Datensatz}\label{subsubsec:datensatz-sequencer}
% https://gitlab.optware.de/sequencer/product/90200438-sequencer-mvp/-/blob/main/libs/interfaces/src/assets/data-generation/generation_input_scheme.json?ref_type=heads
Der Datensatz wurde mittels eines \textit{data generators} künstlich erzeugt.
Dafür wurden für die verschiedenen Parameter Werte ausgewählt bzw.\ ein Bereich angegeben, aus dem der
Wert des Parameters gleichverteilt ausgewählt wurde, um Instanzen verschiedener realer Probleme zu simulieren.

Um die Komplexität des Problems zu verringern, wurden in Bezug auf das oben präsentierte Problem folgende
Vereinfachungen vorgenommen:
\todo[inline]{Liste von Vereinfachungen updaten!}
\begin{itemize}
    \item Jede Station besteht aus nur einem Arbeitsplatz,
    d.h.\ die Aufteilung einer Arbeitsstation in verschiedene Arbeitsplätze wurde weggelassen.
    \item \dots
\end{itemize}
\todo[inline]{Nachdem die Parameter für den Datengenerator festgelegt wurden, sollte hier angegeben werden,
    wie die Parameter gesetzt wurden und wie viele Instanzen welcher Größe generiert wurden.
    Vielleicht kann der generierte Datensatz mit Grafiken/Tabellen dargestellt werden (wie viele Instanzen welcher Art, etc.)?!}
% für folgende Werte kann entweder eine Zahl oder eine Verteilung (number-or-distribution) angegeben werden:
    % cycles
    % numPredecessors
    % numWorkplaces
    % numStations
    % workloadDistribution (pro StationId -> aus dieser Verteilung wird gesamplet, um die processing time pro station zu ermitteln)
% für folgende Werte kann entweder eine Zeiteinheit oder eine Verteilung (time-unit-or-distribution) angegeben werden:
    % maxLateDrift
        % sind keine Drift Parameter angegeben, dann ist an dieser Workstation kein Drift erlaubt
    % cycleTime
%TODO: die Attribute werden automatisch erstellt. An welcher Stelle passiert das? Und wie?
\subsubsection{Anwendungsfall Reihenfolgeplanung (Sequencing)}\label{subsubsec:anwendungsfall-reihenfolgeplanung}
% Allgemeine (high-level) Beschreibung:
Wie bereits in~\cref{subsec:fliessbandfertigung} beschrieben,
folgt in modernen Produktionen auf die Fließbandabstimmung,
in der die benötigten Arbeitsschritte auf Arbeitsstationen verteilt wurden,
häufig die operative Reihenfolge(fein)planung, auch \emph{Sequencing} genannt.
Hierbei geht es nun darum, konkrete Arbeitsaufträge,
die die einzelnen Stationen durchlaufen müssen, in die optimale Reihenfolge zu bringen.
Wir legen hier eine sogenannte \emph{mixed-model assembly line} zugrunde,
was bedeutet, dass die Aufträge die Herstellung verschiedener Modelle eines Basisprodukts beschreiben.
Die benötigten Arbeitsschritte und die Zeit für deren Erledigung unterscheiden sich somit pro Auftrag.
Die operative Feinplanung muss daher u.a.\ dafür sorgen, dass zu keinem Zeitpunkt zu große Abweichungen (Drift)
von den in der Fließbandabstimmung verwendeten Mittelwerten pro Station entstehen.

Eine Probleminstanz für das Sequencing besteht aus
einem Fließband, einer Liste an Aufträgen und einer Liste an Anforderungen bzw.\ Regeln für die Reihenfolge.
Das Ziel ist es, die Aufträge in eine Reihenfolge zu bringen, die diese Regeln beachtet und eine vorher definierte Zielfunktion optimiert.
Im Folgenden werden nun die Eingabeparameter und die verschiedenen Komponenten der Zielfunktion und einer validen Lösung genauer erläutert.

\begin{figure}[H]
    \begin{subfigure}[c]{0.48\textwidth}
        \centering
        \tikzfig{images/reihenfolgeplanung/assembly_line}
        \caption{%
        \label{fig:sequencing_assembly_line}
        Ein Fließband mit drei Stationen,
            die jeweils aus einem Arbeitsplatz bestehen,
            einen Takt dauern
            und eine maximale Driftzeit erlauben.
        Ein Auftrag an Station 1 muss vollständig abgearbeitet sein,
            bevor er an Station 2 weiterbearbeitet werden kann (durchgezogene Linie).
        }
    \end{subfigure}%
    \hfill
    \begin{subfigure}[c]{0.48\textwidth}
        \centering
        \tikzfig{images/reihenfolgeplanung/auftraege}
        \caption{%
        \label{fig:sequencing_auftraege}
        Die Aufträge symbolisieren durch die verschiedenen Formen
        die Herstellung verschiedener Varianten eines Basisprodukts
        und haben eine individuelle Bearbeitungszeit pro Station.
        Die zu erledigenden Aufträge sind vorgegeben.
        }
    \end{subfigure}
    \caption{%
        \label{fig:sequencing_instance}
        Eine einfache Probleminstanz für das Sequencing.
        Das Ziel ist eine optimale Abarbeitungsreihenfolge für die Auftragsliste zu finden.
    }
\end{figure}

\paragraph{Das Fließband}
ist üblicherweise bereits durch die Fließbandabstimmung (siehe~\cref{subsec:anwendungsfall-fliessbandabstimmung})
\marginpar{\scriptsize {\emph{Fließband}:\\ \phantom{s} Taktzeit \\ \phantom{s} Stationen}}
konfiguriert worden und hat somit eine feste Anzahl an \emph{Arbeitsstationen}---denen die verschiedenen Arbeitsschritte,
die an einem Produkt ausgeführt werden müssen, zugeteilt wurden---und eine feste \emph{Taktzeit}.
Ein einfaches Beispiel für ein Fließband mit drei Stationen ist in~\cref{fig:sequencing_assembly_line} illustriert.

Jede Station hat eine vordefinierte Anzahl an \emph{Takten}, für die ein Produkt auf dieser Station verweilen kann,
\marginpar{\scriptsize {\emph{Station}:\\ \phantom{s} Takte \\ \phantom{s} Plätze}}
und ist selbst wieder in eine Menge an \emph{Arbeitsplätze} aufgeteilt.
Jeder Arbeitsplatz kann das Produkt selbst wieder für eine feste Anzahl an Takten bearbeiten
(was auch dazu führen kann, dass sich ein Arbeitsplatz über mehrere Stationen erstreckt)
\marginpar{\scriptsize {\emph{Arbeitsplatz}:\\ \phantom{s} Takte \\ \phantom{s} Vorgänger \\ \phantom{s} Drifts}}
und hat eine Liste an Vorgängerarbeitsplätzen, an denen ein Produkt schon vollständig bearbeitet worden sein muss,
bevor es an dem aktuellen Arbeitsplatz bearbeitet werden kann. % besteht keine Vorgängerbeziehung zwischen zwei Arbeitsplätzen,
% kann also gleichzeitig am Produkt gearbeitet werden (wenn z.B. ein Arbeitsplatz der vorherigen Station über diese hinaus arbeitet)

Des Weiteren können für jeden Arbeitsplatz erlaubte \emph{Drifts} festgelegt werden:
Die Arbeit an diesem Platz darf dann um diese Zeiteinheit früher beginnen bzw.\
über das Ende der Taktzeit hinausdauern.
Bei der ersten Arbeitsstation kann es zudem zu einem initialen Drift kommen,
der z.B.\ durch Verspätungen aus einer vorangegangenen Schicht (mit eigener Reihenfolgeplanung) entstanden ist.
Die Driftgrenzen dürfen nie überschritten werden,
da das Produkt dann beispielsweise außerhalb des Arbeitsradius bestimmter Werkzeuge gelangt und das Fließband gestoppt werden müsste.
Wenn der maximale Drift an einem Arbeitsplatz überschritten würde, muss daher eine sogenannte \emph{Springerin} eingesetzt werden,
die dann pünktlich den Folgeauftrag übernimmt, während die Person an dem Arbeitsplatz ihren aktuellen Auftrag beendet.
Diese Person kann dann den Folgeauftrag, der ja bereits von der Springerin bearbeitet wird,
überspringen und pünktlich zum Taktschlag mit dem nächsten Auftrag fortfahren.

\paragraph{Die Aufträge} bestehen aus einer Auflistung verschiedener Arbeitsschritte, die auf den Arbeitsstationen ausgeführt werden müssen,
\marginpar{\scriptsize {\emph{Auftrag}:\\ \phantom{s} Zeit \\ \phantom{s} Attribute}}
um ein konkretes Produkt herzustellen.
Pro Auftrag kennen wir die benötigte \emph{Arbeitszeit} pro Arbeitsplatz und eine Liste von \emph{Attributen},
anhand derer Regeln für die Reihenfolge bestimmt werden können.
\todo{\tiny Sollen Attribute zunächst vernachlässigt werden? Wenn nicht, sollte das Beispiel ggf. erweitert werden}
Ein vereinfachtes Beispiel ohne Attribute ist in~\cref{fig:sequencing_auftraege} dargestellt.

\paragraph{Die Regeln} schränken den Lösungsraum, also die Anzahl zulässiger Reihenfolgen für die Aufträge,
zusätzlich ein.
So lässt sich beispielsweise festlegen, dass Aufträge mit einem bestimmten Attribut einen gewissen zeitlichen Mindestabstand haben.
Die Regeln stellen jedoch keine starren Vorschriften dar, sondern sollen so gut es geht erfüllt werden.
% ,weshalb sich ihre Nichterfüllung negativ auf den Zielfunktionswert auswirkt.

\paragraph{Die Lösung} einer konkreten Probleminstanz beinhaltet eine hinsichtlich der Zielfunktion optimierte Reihenfolge der Aufträge.
Konkret setzt sich der Zielfunktionswert aus drei Komponenten zusammen, die je nach Wichtigkeit mit einem festgelegten Faktor in die Zielfunktion eingehen:
Die Anzahl an benötigten Springerinnen, der Drift und die Anzahl an verletzen Regeln.
Im einfachen Beispiel (ohne Attribute und Regeln) in~\cref{fig:example_jumper} würden dementsprechend die benötigte Springerin
und die Drifts an Station 1 und 2 in die Zielfunktion eingehen.
\cref{fig:example_optimal} stellt zudem eine Lösung der Instanz dar, die ohne Springerinnen auskommt.

\begin{figure}[H]
    \centering
    \begin{subfigure}[t]{0.48\textwidth}
        \resizebox{1\textwidth}{!}{
            \tikzfig{images/reihenfolgeplanung/example_jumper}
        }
        \caption{%
        \label{fig:example_jumper}
        Die hier angegebene (nicht optimale) Lösung erfordert eine Springerin (rot),
        da der dritte Auftrag sonst den zulässigen maximalen Drift an Station 1 überschritte.
        }
    \end{subfigure}%
    \hfill
    \begin{subfigure}[t]{0.48\textwidth}
        \resizebox{1\textwidth}{!}{
            \tikzfig{images/reihenfolgeplanung/example_optimal}
        }
        \caption{%
        \label{fig:example_optimal}
        Diese Lösung ist hinsichtlich der Anzahl der Springerinnen und der aufsummierten Drifts optimal.
        }
    \end{subfigure}
    \caption{%
    \label{fig:example_sequencing}
    Beispiel für zwei Lösungen der Instanz aus \cref{fig:sequencing_instance}.
    Die festgelegte Auftragsreihenfolge liest sich von oben nach unten
    und die Bearbeitungszeiten der Aufträge sind durch die Länge der horizontalen Balken symbolisiert.
    Zudem markieren die gestrichelten Linien die maximalen Drifts an den Stationen.
    Da die Arbeit an Station 2 erst nach dem Arbeitsende an Station 1 beginnen kann,
    kann ein Drift an Station 1 den Bearbeitungsstart an Station 2 verzögern.
    Der Startzeitpunkt auf Station 2 ergibt sich also aus dem Maximum von dem Ende des Auftrags auf Station 1
    und dem Drift des vorherigen Auftrags auf Station 2.
    Station 3 hingegen hat keine Vorgängerbeziehungen und kann alle Aufträge pünktlich starten,
    weil sich ein Drift auf Station 2 nicht auf Station 3 auswirkt,
    sondern die Arbeiten an diesen Stationen gleichzeitig erledigt werden können.}
\end{figure}

% Input Parameter:
%Eine Probleminstanz für die Reihenfolgeplanung enthält folgende Informationen:
%\begin{itemize}
%    \item Für das Fließband (\textit{line}) gibt es:
%    \begin{itemize}
%        \item eine Liste an (verschiedenen) Arbeitsstationen (\textit{stations}) mit:
%        \begin{itemize}
%            \item Anzahl Takte (\textit{cycles}) %ermöglicht unterschiedliche Taktzeiten für die Stationen trotz globaler cycle time
%            \item einer Liste an (verschiedenen) Arbeitsplätzen (\textit{workplaces}) mit:
%            \begin{itemize}
%                \item Anzahl Takte (\textit{cycles})
%                \item Liste an Vorgängern (\textit{predecessors})
%                \item Informationen über den Drift (\textit{maxEarlyDrift, maxLateDrift, initialDrift})
%            \end{itemize}
%        \end{itemize}
%        \item eine globale Taktzeit (\textit{cycle time}) pro Arbeitsstation (wobei eine Station jedoch mehrere Takte dauern kann)
%    \end{itemize}
%    \item Außerdem gibt es eine Anzahl an Aufträgen (\textit{numOrders}), welche aus verschiedenen Arbeitsschritten,
%    die bei der Fließbandabstimmung (siehe \cref{subsec:anwendungsfall-fliessbandabstimmung}) beachtet wurden und
%    hier nicht mehr relevant sind.
%    \item Für jeden Auftrag (\textit{order}) kennen wir:
%    \begin{itemize}
%        \item die benötigte Arbeitszeit (\textit{processingTime}) an einer gegebenen Arbeitsstation
%        \item eine Liste an Attributen (\textit{Attributes})  % kein input -> automatisch erstellt
%    \end{itemize}
%    \item Zudem gibt es diverse Anforderungen an bzw.\ Regeln für die Reihenfolge
    %https://gitlab.optware.de/sequencer/product/90200438-sequencer-mvp/-/blob/main/libs/interfaces/src/assets/problem/problem_data_schema.json?ref_type=heads
%\end{itemize}

% Output -> result.svg
%Eine Lösung, also eine konkrete Reihenfolgeplanung, enthält
%\begin{itemize}
%    \item eine Reihenfolge der Aufträge (\textit{SequenceResult})
%    \item und einen Zielfunktionswert, der sich zusammensetzt aus:
%    \begin{itemize}
%        \item jumper
%        \item sequencingRules % Anzahl verletzter Regeln
%        \item lateDrift
%    \end{itemize}
%\end{itemize}

\subsubsection*{Datensatz}\label{subsubsec:datensatz-sequencer}
% https://gitlab.optware.de/sequencer/product/90200438-sequencer-mvp/-/blob/main/libs/interfaces/src/assets/data-generation/generation_input_scheme.json?ref_type=heads
Der Datensatz wurde mittels eines \textit{data generators} künstlich erzeugt.
Dafür wurden für die verschiedenen Parameter Werte ausgewählt bzw.\ ein Bereich angegeben, aus dem der
Wert des Parameters gleichverteilt ausgewählt wurde, um Instanzen verschiedener realer Probleme zu simulieren.

Um die Komplexität des Problems zu verringern, wurden in Bezug auf das oben präsentierte Problem folgende
Vereinfachungen vorgenommen:
\todo[inline]{Liste von Vereinfachungen updaten!}
\begin{itemize}
    \item Jede Station besteht aus nur einem Arbeitsplatz,
    d.h.\ die Aufteilung einer Arbeitsstation in verschiedene Arbeitsplätze wurde weggelassen.
    \item \dots
\end{itemize}
\todo[inline]{Nachdem die Parameter für den Datengenerator festgelegt wurden, sollte hier angegeben werden,
    wie die Parameter gesetzt wurden und wie viele Instanzen welcher Größe generiert wurden.
    Vielleicht kann der generierte Datensatz mit Grafiken/Tabellen dargestellt werden (wie viele Instanzen welcher Art, etc.)?!}
% für folgende Werte kann entweder eine Zahl oder eine Verteilung (number-or-distribution) angegeben werden:
    % cycles
    % numPredecessors
    % numWorkplaces
    % numStations
    % workloadDistribution (pro StationId -> aus dieser Verteilung wird gesamplet, um die processing time pro station zu ermitteln)
% für folgende Werte kann entweder eine Zeiteinheit oder eine Verteilung (time-unit-or-distribution) angegeben werden:
    % maxLateDrift
        % sind keine Drift Parameter angegeben, dann ist an dieser Workstation kein Drift erlaubt
    % cycleTime
%TODO: die Attribute werden automatisch erstellt. An welcher Stelle passiert das? Und wie?
\subsubsection{Anwendungsfall Reihenfolgeplanung (Sequencing)}\label{subsubsec:anwendungsfall-reihenfolgeplanung}
% Allgemeine (high-level) Beschreibung:
Wie bereits in~\cref{subsec:fliessbandfertigung} beschrieben,
folgt in modernen Produktionen auf die Fließbandabstimmung,
in der die benötigten Arbeitsschritte auf Arbeitsstationen verteilt wurden,
häufig die operative Reihenfolge(fein)planung, auch \emph{Sequencing} genannt.
Hierbei geht es nun darum, konkrete Arbeitsaufträge,
die die einzelnen Stationen durchlaufen müssen, in die optimale Reihenfolge zu bringen.
Wir legen hier eine sogenannte \emph{mixed-model assembly line} zugrunde,
was bedeutet, dass die Aufträge die Herstellung verschiedener Modelle eines Basisprodukts beschreiben.
Die benötigten Arbeitsschritte und die Zeit für deren Erledigung unterscheiden sich somit pro Auftrag.
Die operative Feinplanung muss daher u.a.\ dafür sorgen, dass zu keinem Zeitpunkt zu große Abweichungen (Drift)
von den in der Fließbandabstimmung verwendeten Mittelwerten pro Station entstehen.

Eine Probleminstanz für das Sequencing besteht aus
einem Fließband, einer Liste an Aufträgen und einer Liste an Anforderungen bzw.\ Regeln für die Reihenfolge.
Das Ziel ist es, die Aufträge in eine Reihenfolge zu bringen, die diese Regeln beachtet und eine vorher definierte Zielfunktion optimiert.
Im Folgenden werden nun die Eingabeparameter und die verschiedenen Komponenten der Zielfunktion und einer validen Lösung genauer erläutert.

\begin{figure}[H]
    \begin{subfigure}[c]{0.48\textwidth}
        \centering
        \tikzfig{images/reihenfolgeplanung/assembly_line}
        \caption{%
        \label{fig:sequencing_assembly_line}
        Ein Fließband mit drei Stationen,
            die jeweils aus einem Arbeitsplatz bestehen,
            einen Takt dauern
            und eine maximale Driftzeit erlauben.
        Ein Auftrag an Station 1 muss vollständig abgearbeitet sein,
            bevor er an Station 2 weiterbearbeitet werden kann (durchgezogene Linie).
        }
    \end{subfigure}%
    \hfill
    \begin{subfigure}[c]{0.48\textwidth}
        \centering
        \tikzfig{images/reihenfolgeplanung/auftraege}
        \caption{%
        \label{fig:sequencing_auftraege}
        Die Aufträge symbolisieren durch die verschiedenen Formen
        die Herstellung verschiedener Varianten eines Basisprodukts
        und haben eine individuelle Bearbeitungszeit pro Station.
        Die zu erledigenden Aufträge sind vorgegeben.
        }
    \end{subfigure}
    \caption{%
        \label{fig:sequencing_instance}
        Eine einfache Probleminstanz für das Sequencing.
        Das Ziel ist eine optimale Abarbeitungsreihenfolge für die Auftragsliste zu finden.
    }
\end{figure}

\paragraph{Das Fließband}
ist üblicherweise bereits durch die Fließbandabstimmung (siehe~\cref{subsec:anwendungsfall-fliessbandabstimmung})
\marginpar{\scriptsize {\emph{Fließband}:\\ \phantom{s} Taktzeit \\ \phantom{s} Stationen}}
konfiguriert worden und hat somit eine feste Anzahl an \emph{Arbeitsstationen}---denen die verschiedenen Arbeitsschritte,
die an einem Produkt ausgeführt werden müssen, zugeteilt wurden---und eine feste \emph{Taktzeit}.
Ein einfaches Beispiel für ein Fließband mit drei Stationen ist in~\cref{fig:sequencing_assembly_line} illustriert.

Jede Station hat eine vordefinierte Anzahl an \emph{Takten}, für die ein Produkt auf dieser Station verweilen kann,
\marginpar{\scriptsize {\emph{Station}:\\ \phantom{s} Takte \\ \phantom{s} Plätze}}
und ist selbst wieder in eine Menge an \emph{Arbeitsplätze} aufgeteilt.
Jeder Arbeitsplatz kann das Produkt selbst wieder für eine feste Anzahl an Takten bearbeiten
(was auch dazu führen kann, dass sich ein Arbeitsplatz über mehrere Stationen erstreckt)
\marginpar{\scriptsize {\emph{Arbeitsplatz}:\\ \phantom{s} Takte \\ \phantom{s} Vorgänger \\ \phantom{s} Drifts}}
und hat eine Liste an Vorgängerarbeitsplätzen, an denen ein Produkt schon vollständig bearbeitet worden sein muss,
bevor es an dem aktuellen Arbeitsplatz bearbeitet werden kann. % besteht keine Vorgängerbeziehung zwischen zwei Arbeitsplätzen,
% kann also gleichzeitig am Produkt gearbeitet werden (wenn z.B. ein Arbeitsplatz der vorherigen Station über diese hinaus arbeitet)

Des Weiteren können für jeden Arbeitsplatz erlaubte \emph{Drifts} festgelegt werden:
Die Arbeit an diesem Platz darf dann um diese Zeiteinheit früher beginnen bzw.\
über das Ende der Taktzeit hinausdauern.
Bei der ersten Arbeitsstation kann es zudem zu einem initialen Drift kommen,
der z.B.\ durch Verspätungen aus einer vorangegangenen Schicht (mit eigener Reihenfolgeplanung) entstanden ist.
Die Driftgrenzen dürfen nie überschritten werden,
da das Produkt dann beispielsweise außerhalb des Arbeitsradius bestimmter Werkzeuge gelangt und das Fließband gestoppt werden müsste.
Wenn der maximale Drift an einem Arbeitsplatz überschritten würde, muss daher eine sogenannte \emph{Springerin} eingesetzt werden,
die dann pünktlich den Folgeauftrag übernimmt, während die Person an dem Arbeitsplatz ihren aktuellen Auftrag beendet.
Diese Person kann dann den Folgeauftrag, der ja bereits von der Springerin bearbeitet wird,
überspringen und pünktlich zum Taktschlag mit dem nächsten Auftrag fortfahren.

\paragraph{Die Aufträge} bestehen aus einer Auflistung verschiedener Arbeitsschritte, die auf den Arbeitsstationen ausgeführt werden müssen,
\marginpar{\scriptsize {\emph{Auftrag}:\\ \phantom{s} Zeit \\ \phantom{s} Attribute}}
um ein konkretes Produkt herzustellen.
Pro Auftrag kennen wir die benötigte \emph{Arbeitszeit} pro Arbeitsplatz und eine Liste von \emph{Attributen},
anhand derer Regeln für die Reihenfolge bestimmt werden können.
\todo{\tiny Sollen Attribute zunächst vernachlässigt werden? Wenn nicht, sollte das Beispiel ggf. erweitert werden}
Ein vereinfachtes Beispiel ohne Attribute ist in~\cref{fig:sequencing_auftraege} dargestellt.

\paragraph{Die Regeln} schränken den Lösungsraum, also die Anzahl zulässiger Reihenfolgen für die Aufträge,
zusätzlich ein.
So lässt sich beispielsweise festlegen, dass Aufträge mit einem bestimmten Attribut einen gewissen zeitlichen Mindestabstand haben.
Die Regeln stellen jedoch keine starren Vorschriften dar, sondern sollen so gut es geht erfüllt werden.
% ,weshalb sich ihre Nichterfüllung negativ auf den Zielfunktionswert auswirkt.

\paragraph{Die Lösung} einer konkreten Probleminstanz beinhaltet eine hinsichtlich der Zielfunktion optimierte Reihenfolge der Aufträge.
Konkret setzt sich der Zielfunktionswert aus drei Komponenten zusammen, die je nach Wichtigkeit mit einem festgelegten Faktor in die Zielfunktion eingehen:
Die Anzahl an benötigten Springerinnen, der Drift und die Anzahl an verletzen Regeln.
Im einfachen Beispiel (ohne Attribute und Regeln) in~\cref{fig:example_jumper} würden dementsprechend die benötigte Springerin
und die Drifts an Station 1 und 2 in die Zielfunktion eingehen.
\cref{fig:example_optimal} stellt zudem eine Lösung der Instanz dar, die ohne Springerinnen auskommt.

\begin{figure}[H]
    \centering
    \begin{subfigure}[t]{0.48\textwidth}
        \resizebox{1\textwidth}{!}{
            \tikzfig{images/reihenfolgeplanung/example_jumper}
        }
        \caption{%
        \label{fig:example_jumper}
        Die hier angegebene (nicht optimale) Lösung erfordert eine Springerin (rot),
        da der dritte Auftrag sonst den zulässigen maximalen Drift an Station 1 überschritte.
        }
    \end{subfigure}%
    \hfill
    \begin{subfigure}[t]{0.48\textwidth}
        \resizebox{1\textwidth}{!}{
            \tikzfig{images/reihenfolgeplanung/example_optimal}
        }
        \caption{%
        \label{fig:example_optimal}
        Diese Lösung ist hinsichtlich der Anzahl der Springerinnen und der aufsummierten Drifts optimal.
        }
    \end{subfigure}
    \caption{%
    \label{fig:example_sequencing}
    Beispiel für zwei Lösungen der Instanz aus \cref{fig:sequencing_instance}.
    Die festgelegte Auftragsreihenfolge liest sich von oben nach unten
    und die Bearbeitungszeiten der Aufträge sind durch die Länge der horizontalen Balken symbolisiert.
    Zudem markieren die gestrichelten Linien die maximalen Drifts an den Stationen.
    Da die Arbeit an Station 2 erst nach dem Arbeitsende an Station 1 beginnen kann,
    kann ein Drift an Station 1 den Bearbeitungsstart an Station 2 verzögern.
    Der Startzeitpunkt auf Station 2 ergibt sich also aus dem Maximum von dem Ende des Auftrags auf Station 1
    und dem Drift des vorherigen Auftrags auf Station 2.
    Station 3 hingegen hat keine Vorgängerbeziehungen und kann alle Aufträge pünktlich starten,
    weil sich ein Drift auf Station 2 nicht auf Station 3 auswirkt,
    sondern die Arbeiten an diesen Stationen gleichzeitig erledigt werden können.}
\end{figure}

% Input Parameter:
%Eine Probleminstanz für die Reihenfolgeplanung enthält folgende Informationen:
%\begin{itemize}
%    \item Für das Fließband (\textit{line}) gibt es:
%    \begin{itemize}
%        \item eine Liste an (verschiedenen) Arbeitsstationen (\textit{stations}) mit:
%        \begin{itemize}
%            \item Anzahl Takte (\textit{cycles}) %ermöglicht unterschiedliche Taktzeiten für die Stationen trotz globaler cycle time
%            \item einer Liste an (verschiedenen) Arbeitsplätzen (\textit{workplaces}) mit:
%            \begin{itemize}
%                \item Anzahl Takte (\textit{cycles})
%                \item Liste an Vorgängern (\textit{predecessors})
%                \item Informationen über den Drift (\textit{maxEarlyDrift, maxLateDrift, initialDrift})
%            \end{itemize}
%        \end{itemize}
%        \item eine globale Taktzeit (\textit{cycle time}) pro Arbeitsstation (wobei eine Station jedoch mehrere Takte dauern kann)
%    \end{itemize}
%    \item Außerdem gibt es eine Anzahl an Aufträgen (\textit{numOrders}), welche aus verschiedenen Arbeitsschritten,
%    die bei der Fließbandabstimmung (siehe \cref{subsec:anwendungsfall-fliessbandabstimmung}) beachtet wurden und
%    hier nicht mehr relevant sind.
%    \item Für jeden Auftrag (\textit{order}) kennen wir:
%    \begin{itemize}
%        \item die benötigte Arbeitszeit (\textit{processingTime}) an einer gegebenen Arbeitsstation
%        \item eine Liste an Attributen (\textit{Attributes})  % kein input -> automatisch erstellt
%    \end{itemize}
%    \item Zudem gibt es diverse Anforderungen an bzw.\ Regeln für die Reihenfolge
    %https://gitlab.optware.de/sequencer/product/90200438-sequencer-mvp/-/blob/main/libs/interfaces/src/assets/problem/problem_data_schema.json?ref_type=heads
%\end{itemize}

% Output -> result.svg
%Eine Lösung, also eine konkrete Reihenfolgeplanung, enthält
%\begin{itemize}
%    \item eine Reihenfolge der Aufträge (\textit{SequenceResult})
%    \item und einen Zielfunktionswert, der sich zusammensetzt aus:
%    \begin{itemize}
%        \item jumper
%        \item sequencingRules % Anzahl verletzter Regeln
%        \item lateDrift
%    \end{itemize}
%\end{itemize}

\subsubsection*{Datensatz}\label{subsubsec:datensatz-sequencer}
% https://gitlab.optware.de/sequencer/product/90200438-sequencer-mvp/-/blob/main/libs/interfaces/src/assets/data-generation/generation_input_scheme.json?ref_type=heads
Der Datensatz wurde mittels eines \textit{data generators} künstlich erzeugt.
Dafür wurden für die verschiedenen Parameter Werte ausgewählt bzw.\ ein Bereich angegeben, aus dem der
Wert des Parameters gleichverteilt ausgewählt wurde, um Instanzen verschiedener realer Probleme zu simulieren.

Um die Komplexität des Problems zu verringern, wurden in Bezug auf das oben präsentierte Problem folgende
Vereinfachungen vorgenommen:
\todo[inline]{Liste von Vereinfachungen updaten!}
\begin{itemize}
    \item Jede Station besteht aus nur einem Arbeitsplatz,
    d.h.\ die Aufteilung einer Arbeitsstation in verschiedene Arbeitsplätze wurde weggelassen.
    \item \dots
\end{itemize}
\todo[inline]{Nachdem die Parameter für den Datengenerator festgelegt wurden, sollte hier angegeben werden,
    wie die Parameter gesetzt wurden und wie viele Instanzen welcher Größe generiert wurden.
    Vielleicht kann der generierte Datensatz mit Grafiken/Tabellen dargestellt werden (wie viele Instanzen welcher Art, etc.)?!}
% für folgende Werte kann entweder eine Zahl oder eine Verteilung (number-or-distribution) angegeben werden:
    % cycles
    % numPredecessors
    % numWorkplaces
    % numStations
    % workloadDistribution (pro StationId -> aus dieser Verteilung wird gesamplet, um die processing time pro station zu ermitteln)
% für folgende Werte kann entweder eine Zeiteinheit oder eine Verteilung (time-unit-or-distribution) angegeben werden:
    % maxLateDrift
        % sind keine Drift Parameter angegeben, dann ist an dieser Workstation kein Drift erlaubt
    % cycleTime
%TODO: die Attribute werden automatisch erstellt. An welcher Stelle passiert das? Und wie?

    %--------------------------------------

    % Glossar
    %--------------------------------------
    \newpage
    %\printglossary
    %--------------------------------------

    % Version History
    %--------------------------------------
    \newpage
    \begin{versionhistory}
        %Example usage
        %\vhEntry{VERSION}{DATE}{AUTHOR(S)}{DESCRPTION}
        \vhEntry{0.0}{2023-07-28}{OptWare}{Initiale Version: Beschreibung UseCase, Bin Packing}
        \vhEntry{0.1}{2023-09-05}{OptWare}{Anpassung Beschreibung Testfahrzeugkonfiguration}
        \vhEntry{0.2}{2023-09-20}{OptWare}{Anpassung der Zielfunktionen}
        \vhEntry{0.3}{2023-12-19}{OptWare}{Beschreibung des Andwenungsfalls - Reihenfolgeplanung}
    \end{versionhistory}
    %--------------------------------------

    % Literatur
    %--------------------------------------
    \newpage
    \printbibliography
    %--------------------------------------


    % List of todos
    %--------------------------------------
    \newpage
    \todototoc
    \listoftodos
    \newpage
    %--------------------------------------

    % Appendix
    %--------------------------------------
    \begin{appendices}
    \end{appendices}
    %--------------------------------------
\end{document}
