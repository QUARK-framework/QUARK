\subsubsection{Anwendungsfall Fließbandabstimmung}\label{subsec:anwendungsfall-fliessbandabstimmung}

\begin{figure}[h]
    \centering
    \tikzfig{images/fliessbandabstimmung/balancing}
    \caption{%
    \label{fig:balancing}
    Beispiel für Fließbandabstimmung: Die einem Reihenfolgegraph unterliegenden Arbeitsschritte (1-6)
        sollen nach Zeitbedarf gleichmäßig auf Arbeitsstationen (I-IV) verteilt werden.
        Dabei darf die Arbeitslast pro Station 25 Sekunden nicht überschreiten.}
\end{figure}

Gegeben ist bei dem Problem der Fließbandabstimmung eine Reihe von Arbeitsschritten, die Arbeitsstationen zugeordnet werden sollen.
Dabei soll eine möglichst geringe Anzahl und eine im Mittel gleichmäßige Auslastung der Arbeitsstationen angestrebt werden.
Pro Arbeitsstation steht maximal ein Takt zur Verfügung. Zum Taktschlag sollen die Arbeitsaufträge auf der nächsten Station bearbeitet werden.
Somit darf die summierte durchschnittliche Arbeitszeit der Arbeitsschritte die Taktzeit nicht überschreiten.

Die Arbeitsschritte müssen teilweise in einer vorgegebenen Reihenfolge abgearbeitet werden.
\Cref{fig:balancing} zeigt im oberen Teil einen beispielhaften Reihenfolgegraphen
mit den Tätigkeiten (1-6) zu denen auch jeweils eine mittlere Dauer
(z.B. 10 Sekunden bei Tätigkeit 1) angegeben ist.
Tätigkeit 2 hängt von Tätigkeit 1 ab und kann deshalb erst durchgeführt werden,
wenn Tätigkeit 1 bereits abgeschlossen ist.
Gleiches gilt für Tätigkeit 3, sie hängt wiederum von Tätigkeit 2 ab.
Tätigkeit 4 im Beispiel kann parallel zu Tätigkeit 2 erfolgen,
da sie nur von Tätigkeit 1 abhängig ist und nicht etwa von Tätigkeit 2 oder 3.

In \cref{fig:balancing} sind im unteren Teil Arbeitsstationen (I-IV) als offene Behälter angedeutet.
Den Arbeitsstationen sollen nun, unter Achtung des Reihenfolgegraphen, Tätigkeiten zugeteilt werden.
Die Taktzeit beträgt 25 Sekunden. Die Zuordnung in \cref{fig:solution_ok} wäre denkbar,
während die Zuordnung in \cref{fig:solution_fail} nicht möglich ist, da sie dem
Reihenfolgegraphen widerspricht und Station III die Taktzeit überschreitet.

\begin{figure}[h]
    \centering
    \tikzfig{images/fliessbandabstimmung/solution_ok}
    \caption{Mögliche Lösung für das Ausgangsproblem aus \cref{fig:balancing}}
    \label{fig:solution_ok}
\end{figure}
\begin{figure}[h]
    \centering
    \tikzfig{images/fliessbandabstimmung/solution_fail}
    \caption{Unzulässige Lösung für das Ausgangsproblem aus \cref{fig:balancing}}
    \label{fig:solution_fail}
\end{figure}

\subsubsection*{Transformation zu Bin Packing}
% \todo[inline]{evtl. ein eigenes Kapitel für das Mapping einfügen.}
Das Fließbandabstimmungsproblem lässt sich klassisch als Assembly Line Balancing Problem aus
\cref{subsubsec:assembly-line-balancing-problem} aufstellen und lösen.
Die Arbeitsschritte sind dabei die Aufgaben und die Arbeitsstationen die Stationen des klassischen Problems.
Die Taktzeit wird durch die maximale Zeit pro Station dargestellt.

Zusammengefasst, lässt sich die Fließbandabstimmung wie folgt auf die Notation aus
\cref{subsubsec:assembly-line-balancing-problem} mappen:
\begin{table}[H]
    \begin{tabularx}{\textwidth}{  l | X | l }
    Symbol & Bedeutung & Definitionsbereich \\\hline\hline
    Mengen & & \\\hline\hline
    $T$ & Menge der Arbeitsschritte & $\N$\\\hline
    $S$ & Menge der Arbeitsstationen & $\N$\\\hline
    $R$ & Menge an Abhängigkeitsbeziehungen zwischen den Arbeitsschritten & $\N\times\N$\\\hline\hline
    Konstanten &  &  \\\hline\hline
    $c$ & Taktzeit & $\N$\\\hline
    $v_t$ & Benötigte Zeit für Arbeitsschritt $t\in T$ & $\N$\\\hline\hline
    Variablen &  &  \\\hline\hline
    $x_{ts}$ & Genau dann, wenn $x_{ts}=1$ wird Arbeitsschritt $t\in T$ der Station $s\in S$ zugeteilt & $\B$\\\hline
    \end{tabularx}
    \caption{Notation Fließbandabstimmung}\label{tab:notification_fliessbandabstimmung}
\end{table}
