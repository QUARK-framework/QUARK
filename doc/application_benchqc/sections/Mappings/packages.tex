% Mehr Operatoren und Symbolen, verschiedene Theoremumgebungen
% und Darstellungsoptionen
\usepackage{amsmath,amssymb,amsthm,amsfonts}
\usepackage{eurosym}
\usepackage{mathtools}
\mathtoolsset{showonlyrefs}
\usepackage{color}
\usepackage{nicefrac}       % compact symbols for 1/2, etc.
\usepackage{rotating} % Rotating table
% Zum besseren Definieren von Abkürzungen
\usepackage{xspace}

% Float-Paket, H-Befehl für Tabellen, schöneres Tabellendesign
\usepackage{float}
\restylefloat{table}
\usepackage{booktabs}

% Grafiken
\usepackage{graphicx}
\usepackage{color}
\usepackage{epstopdf}
\DeclareGraphicsExtensions{.pdf,.eps,.png}
\usepackage{subfig}
\usepackage{stackengine}
\usepackage{tikz}
\usetikzlibrary{shapes.misc}

% Adressenangabe der Autoren per Fußnote
%\usepackage{authblk}


% Erweiterte Optionen für Aufzählungen
\usepackage{paralist}
\usepackage{enumitem}


% Schönere Symbole für Zahlenräume
\usepackage{bbm}

% Hyperlinks im Dokument und PDF-Optionen
%\usepackage[bookmarksnumbered,bookmarksopen,unicode]{hyperref}
%\hypersetup{
%	pdftitle={},
%	pdfauthor={},
%	pdfkeywords={},% das gleiche wie be \keywords
%	pdfsubject={MSC2000 Primary ;Secondary} % "MSC2000" + das gleiche wie bei \subjclass{}
%}

% Berechnen von Größen
\usepackage{calc}

% Akzente unterhalb der Variablen
\usepackage{accents}

% Notizen
%\usepackage[disable]{todonotes}
%\presetkeys{todonotes}{inline}{}
