\subsection{Bin Packing}\label{subsec:bin-packing}

Beim Bin Packing Problem (oder auch Behälterproblem) geht es darum, eine gegebene Anzahl an Objekten auf eine
kleinstmögliche Anzahl an Behältern zu verteilen, sodass kein Behälter überläuft.

Als ganzzahliges Optimierungsproblem lässt sich das Bin Packing Problem wie folgt darstellen:
Jedes Objekt muss genau einem Behälter zugeteilt werden~\eqref{eq:one_bin_per_object}.
Pro Behälter darf die maximale Kapazität nicht überschritten werden~\eqref{eq:max_capacity_bin}.
Das Ziel ist, die Anzahl der Behälter zu minimieren, was mithilfe der Zielfunktion~\eqref{eq:minimize_bins} erreicht wird.

%{Die Zielfunktion minimiert die Anzahl der Behälter nicht immer. In der Literatur werden andere
%Formulierungen genutzt, die jedoch mehr binäre Variablen benötigen. Daher sollte die Wahl der Zielfunktion
%noch weiter diskutiert werden.}

\begin{table}[H]
    \begin{tabularx}{\textwidth}{  l | X | l }
    Symbol & Bedeutung & Definitionsbereich \\\hline\hline
    Mengen & & \\\hline\hline
    $O$ & Menge der Objekte & $\N$\\\hline
    $B$ & Menge der Behälter & $\N$\\\hline\hline
    Konstanten &  &  \\\hline\hline
    $c$ & Maximale Kapazität der Behälter & $\N$\\\hline
    $w_o$ & Gewicht von Objekt $o\in O$ & $\N$\\\hline\hline
    Variablen &  &  \\\hline\hline
    $x_{ob}$ & $x_{ob}=1$ gdw.\ Objekt $o\in O$ dem Behälter $b\in B$ zugeteilt wird & $\B$\\\hline
    $y_b$ & $y_b=1$ gdw.\ Behälter $b\in B$ verwendet wird & $\B$\\\hline
    \end{tabularx}
    \caption{Notation Bin Packing}\label{tab:notation_bin_packing}
\end{table}

%\begin{align}
%    \min \qquad& \sum_{b\in B,\,o\in O} b \cdot x_{ob} \label{eq:minimize_bins}\\
%    \textit{s.t.} \qquad
%    & \sum_{b\in B} x_{ob} = 1~, \qquad &&\forall o\in O \label{eq:one_bin_per_object} \\
%    & \sum_{o\in O} w_{o} \cdot x_{ob} \leq c~, \qquad &&\forall b\in B \label{eq:max_capacity_bin}\\
%    & x_{ob} \in \B~, \qquad &&\forall o\in O, b\in B
%\end{align}

\begin{align}
    \min \qquad& \sum_{b\in B} y_b \label{eq:minimize_bins}\\
    \textit{s.t.} \qquad
    & \sum_{b\in B} x_{ob} = 1~, \qquad &&\forall o\in O \label{eq:one_bin_per_object} \\
    & \sum_{o\in O} w_{o} \cdot x_{ob} \leq c \cdot y_b~, \qquad &&\forall b\in B \label{eq:max_capacity_bin}\\
    & x_{ob}, y_b \in \B~, \qquad &&\forall o\in O, b\in B
\end{align}

\paragraph{Alternative Zielfunktionen.}
% K: eine weitere interessante Formulierung mit wenigen Variablen findet sich in: https://hal.science/hal-03006106/document
% bisher habe ich aber noch keine Möglichkeit gefunden, in dieser Formulierung Precedence Relations einzubauen.
% Daher habe ich sie hier noch nicht genannt.

In einer früheren Version dieses Dokuments wurde als Zielfunktion statt ~\eqref{eq:minimize_bins}
\[
    \sum_{b\in B,\,o\in O} b \cdot x_{ob}
\]
minimiert,\footnote{Die Constraints können fast unverändert bleiben; es müssen lediglich die $y$ Variablen gelöscht werden.}
was auch in der Literatur (für das speziellere Assembly Line Balancing Problem) zu finden ist~\cite{alb_paper_50}.
Diese Version ermöglicht im Vergleich zu der obigen Formulierung zwar die $y$ Variablen einzusparen,
führt aufgrund der Gewichtung allerdings manchmal zu suboptimalen Lösungen in Bezug auf die Anzahl der benötigten Bins.

Um dieses Problem zu vermeiden, kann in der Zielfunktion mit einer \emph{cost explosion} gearbeitet werden,
um zusätzliche Bins stärker zu bestrafen:
\[
    \min \sum_{b\in B,\,o\in O} |O|^{b} \cdot x_{ob}
\]
\todo[inline]{Der Exponent kann auch $b-1$ sein, wenn Bin $0$ den Koeffizienten $0$ hat.}
Da die Koeffizienten, abhängig von $|B|$, sehr groß werden können, kommt es in der Praxis jedoch schnell zu numerischen Problemen,
weshalb diese Variante hier nur von theoretischem Interesse ist.
